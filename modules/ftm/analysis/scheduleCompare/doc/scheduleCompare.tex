\documentclass[12pt,a4paper]{report}
\pdfminorversion=7
\usepackage[pdftex]{graphicx}
\usepackage{tabularx, multirow}
\usepackage{amssymb}
\begin{document}
\begin{titlepage}
\begin{center}
\vspace{2em}
\Huge{ScheduleCompare}\\[2cm]
\Large{Documentation}\\[2cm]
\begin{large}
\begin{tabularx}{\textwidth}{Xl}
Version & 1.0\\
Last updated & 2021-11-23\\
\vspace{1.5cm}\\
Author & Martin Skorsky\\
Department & ACO \\
Group & TOS\\
Contact & Martin Skorsky
\end{tabularx}%
\end{large}

\vfill

\end{center}
\end{titlepage}
\tableofcontents
\chapter{Purpose of ScheduleCompare}
The tool \texttt{scheduleCompare} tests if two graphs describe the same schedule. The graphs are written as dot-files.

The output is either "graphs represent the same schedule" or "graphs represent different schedules and the differences are \textellipsis".
The output in the later case should show at least the first vertex with non-matching attributes.
It is also interesting to know that $g_1$ is a subgraph of $g_2$. This is also visible in the output.

The algorithm used ist the method \texttt{vf2\_subgraph\_iso} from the Boost Graph Library. This tests if a graph $g_1$ is isomorph
to a subgraph og graph$g_2$. Thus, we have to test that $g_1$ is a subgraph of $g_2$ and $g_2$ is a subgraph $g_1$ to check
that the two graphs are isomorph. This test can be shortened by testing that $g_1$ is a subgraph of $g_2$ and the number
of vertices are equal $|V(g_1)| = |V(g_2)|$ and the number of edges are equal $|E(g_1)| = |E(g_2)|$.

In the following table, $g_1 S g_2$
means $g_1$ is possibly a subgraph of $g_2$. In addition, $g_1 S g_2$ means $g_1$ is possibly isomorphic to $g_2$.
\begin{table}
\label{tab:cardinalities}
\begin{tabular}[t]{|c|ccc|}
\hline
                     & $|E(g_1)|<|E(g_2)|$ & $|E(g_1)|=|E(g_2)|$ & $|E(g_1)|>|E(g_2)|$ \\ \hline
 $|V(g_1)|<|V(g_2)|$ &          S          &          S          &          -  \\
 $|V(g_1)|=|V(g_2)|$ &          S          &          I          &          -  \\
 $|V(g_1)|>|V(g_2)|$ &          -          &          -          &          -  \\ \hline
\end{tabular}
\end{table}

\section{Prerequisites}
The dot files to compare should be syntactically correct.
They should contain a graph with at least one vertex.
It is not necessary that the dot file can be loaded into the datamaster. This is not checked.
\texttt{libcarpedm} is not needed.

\section{Compare Vertices}
The Boost Graph algorithm needs a comparator for the vertices and a second comparator for the edges.
The comparator for the edges is easy: two edges are equal if and only if the types are equal (as strings).

The comparator for the vertices has to handle the class hierarchy to the vertices (subclasses of \texttt{Node} in \texttt{libcarpedm}).
The attributes for vertices are defined in \texttt{libcarpedm} for the dot parser, the members of \texttt{mxVertex}, and the subclasses of \texttt{Node}.
The attributes for \texttt{Node} (see~\ref{tab:Node-attributes}) are common to all vertices.

The comparator for the vertices checks as a first step the types. If a vertex has no type, this is interpreted as a wildcard for a type defined elsewhere.
This occures for dot files with describe a pattern to remove. The dot file to remove describes a schedule which is isomorphic to a subgraph of the larger schedule.

The comparator for the vertices handle sthe different data types of the values. For boolean values, \texttt{0}, \texttt{false}, \texttt{False}, and none (empty string) are interpreted as false.
\texttt{1}, \texttt{true}, and \texttt{True} are interpreted as true. For hexadecimal values, leading zeros between prefix '0x' and the value are removed before comparing.

\begin{table}
\caption{Attributes for \texttt{Node}}
\label{tab:Node-attributes}
\begin{center}
\begin{tabular}[t]{|l|clll|}
\hline
Attribute & Required   & Value Type & Value Range                  & Remark           \\ \hline
name      & \checkmark & String     & vertex name                  &                  \\
pattern   & tested     & String     & pattern name                 &                  \\
patentry  & \checkmark & boolean    & 0,1,True, true, False, false &                  \\
patexit   & \checkmark & boolean    & 0,1,True, true, False, false &                  \\
beamproc  & tested     & String     & beam process name            &                  \\
bpentry   & \checkmark & boolean    & 0,1,True, true, False, false &                  \\
bpexit    & \checkmark & boolean    & 0,1,True, true, False, false &                  \\
cpu       & \checkmark & uint8      & 0,1,2,3                      &                  \\
thread    & not tested & uint8      & 0..7                         & not a parser tag \\
flags     & tested     & uint32     & bit field                    &                  \\
hash      & not tested & uint32     & -                            & not a parser tag \\
\hline
\end{tabular}
\end{center}
\end{table}

The attributes for \texttt{Block} include attributes for \texttt{Node}.
\begin{table}
\caption{Additional attributes for \texttt{Block}}
\label{tab:Block-attributes}
\begin{center}
\begin{tabular}[t]{|l|cll|}
\hline
Attribute & Required   & Value Type & Value Range \\ \hline
tperiod   & \checkmark & uint64     & -           \\
rdidxlo   & not tested & uint8      &             \\
rdidxhi   & not tested & uint8      &             \\
rdidxil   & not tested & uint8      &             \\
wridxlo   & not tested & uint8      &             \\
wridxhi   & not tested & uint8      &             \\
wridxil   & not tested & uint8      &             \\
\hline
\end{tabular}
\end{center}
\end{table}

The attributes for \texttt{Event} include attributes for \texttt{Node}.
\begin{table}
\caption{Additional attributes for \texttt{Event}}
\label{tab:Event-attributes}
\begin{center}
\begin{tabular}[t]{|l|cll|}
\hline
Attribute & Required   & Value Type & Value Range \\ \hline
toffs     & \checkmark & unint64    &             \\
\hline
\end{tabular}
\end{center}
\end{table}

The attributes for \texttt{TimingMsg} include attributes for \texttt{Event} and \texttt{Node}.
\begin{table}
\caption{Additional attributes for \texttt{TimingMsg}}
\label{tab:TimingMsg-attributes}
\begin{center}
\begin{tabular}[t]{|l|cll|}
\hline
Attribute & Required   & Value Type & Value Range \\ \hline
id        & \checkmark & uint64     &             \\
fid       & \checkmark & 4 bit      & 0,1         \\
gid       & \checkmark & 12 bit     &             \\
evtno     & \checkmark & 12 bit     &             \\
bin       & \checkmark & 1 bit      &             \\
bpcstart  & \checkmark & 1 bit      &             \\
res\_1    & not tested & 2 bit      &             \\
sid       & \checkmark & 12 bit     &             \\
bpid      & \checkmark & 14 bit     &             \\
res\_2    & not tested & 1 bit      &             \\
reqnob    & \checkmark & 1 bit      &             \\
vacc      & \checkmark & 4 bit      &             \\
par       & \checkmark & uint64     &             \\
tef       & tested     & uint32     &             \\
res       & tested     & uint32     &             \\
\hline
\end{tabular}
\end{center}
\end{table}

The attributes for \texttt{Command} include attributes for \texttt{Node}.
\begin{table}
\caption{Additional attributes for \texttt{Command}}
\label{tab:Command-attributes}
\begin{center}
\begin{tabular}[t]{|l|cll|}
\hline
Attribute & Required   & Value Type & Value Range \\ \hline
tvalid    & \checkmark & uint64     &             \\
vabs      & tested     & boolean    & 0,1,Ture, true, False, false \\
act       & not tested & unit32     &             \\
target    & \checkmark & String     & vertex name \\
dest      & \checkmark & String     & vertex name \\
\hline
\end{tabular}
\end{center}
\end{table}

The attributes for \texttt{Flush} include attributes for \texttt{Command} and \texttt{Node}.
\begin{table}
\caption{Additional attributes for \texttt{Flush}}
\label{tab:Flush-attributes}
\begin{center}
\begin{tabular}[t]{|l|cll|}
\hline
Attribute & Required   & Value Type & Value Range \\ \hline
mode      &            & uint8      &             \\
qLo       &            &            &             \\
qHi       &            &            &             \\
qIl       &            &            &             \\
frmIl     & not tested & uint8      &             \\
toIl      & not tested & uint8      &             \\
frmHi     & not tested & uint8      &             \\
toHi      & not tested & uint8      &             \\
frmLo     & not tested & uint8      &             \\
toLo      & not tested & uint8      &             \\
\hline
\end{tabular}
\end{center}
\end{table}

The attributes for \texttt{Wait} include attributes for \texttt{Command} and \texttt{Node}.
\begin{table}
\caption{Additional attributes for \texttt{Wait}}
\label{tab:Wait-attributes}
\begin{center}
\begin{tabular}[t]{|l|cll|}
\hline
Attribute & Required   & Value Type & Value Range \\ \hline
twait     & \checkmark & uint64     &             \\
\hline
\end{tabular}
\end{center}
\end{table}

\begin{figure}
\begin{center}
\includegraphics*[height=.93\textheight,keepaspectratio]{attributeCompareTree.pdf}
\caption{Decision Tree to Test Attributes and Types for Vertices}
\label{fig:attributeCompareTree}
\end{center}
\end{figure}

\chapter{Arguments and Options}
Usage: \texttt{scheduleCompare $<$dot file 1$>$ $<$dot file 2$>$}

Options:
\begin{enumerate}
  \item -c: check dot syntax (stops parsing on all unknown attributes).
  \item -h: help and usage.
  \item -n: do not compare names of vertices. Not applicable with option -t.
  \item -s: silent, no output, only return code. Usefull for automated tests.
  \item -t: test a single graph: compare each vertex with itself. This tests the vertex comparator.
  \item -v: verbose, output of input graphs as dot text.
  \item -vv: super verbose, in addition to verbose more output.
\end{enumerate}

Return Codes:
\begin{enumerate}
  \item 0 EXIT\_SUCCESS, graphs are isomorphic.
  \item 1 NOT\_ISOMORPHIC, graphs are not isomorphic.
  \item 2 SUBGRAPH\_ISOMORPHIC, graph is isomorphic to a subgraph of the larger graph.
  \item 11 BAD\_ARGUMENTS, unknown arguments on command line.
  \item 12 MISSING\_ARGUMENT, at least one of the file names is missing.
  \item 13 FILE\_NOT\_FOUND, one of the dot files not found.
  \item 14 USAGE\_MESSAGE, usage message (help) displayed.
  \item 15 PARSE\_ERROR, error while parsing, unknown tag or attribute.
  \item 16 PARSE\_ERROR\_GRAPHVIZ, error while parsing Graphviz syntax.
  \item 17 TEST\_SUCCESS, test a single graph with success.
  \item 18 TEST\_FAIL, test a single graph with failure.
  \item negative values are UNIX signals
\end{enumerate}

\chapter{Testframework and Test Cases}
First attempt for the test framework are unit tests with Python. A test case in general calls \texttt{scheduleCompare} with two dot files and examines the output. For this, the return code of \texttt{scheduleCompare}.

\begin{enumerate}
  \item 9 test cases for the cardinalities from table~\ref{tab:cardinalities}
  \item 1815 dot files (11 * 11 *15) from \texttt{singleEdgeTest}. Each two of these are compared. Only if a dot file is compared with itself, the graphs should be isomorphic. These are 3.294.225 combinations.
This test runs for about 8 hours. Start with \texttt{nohup} $<$ command $>$ \& to run this in the background without a tty / terminal.
  \item same graph, but different names for the vertices.
  \item same graph, but different types for vertices with the same name.
  \item same graph, but different types for edges between the same vertices. Vertex names and types are the same.
  \item test cases for boolean values.
  \item cases for hex values.
\end{enumerate}

\section{Test cases for the protocol}
If the two schedules are not isomorphic, a protocol of the comparison is printed in verbose mode. The test cases
ensure that the protocol works for different situations. These are:
\begin{enumerate}
  \item Set of vertices of graph 1 and graph 2 have the same size.
  \item Set of edges of graph 1 and graph 2 have the same size.
  \item Graph 1 is isomorphic to graph 2.
  \item Graph 1 is isomorphic to a subgraph of graph 2.
  \item Graph 1 is not isomorphic to a subgraph of graph 2 because of vertex comparison.
  \item Graph 1 is not isomorphic to a subgraph of graph 2 because of edge comparison.
  \item Vertex names are equal.
  \item Vertex names are not equal.
\end{enumerate}

List of test cases:
\begin{enumerate}
  \item Graph 1 and graph 2 have the same size and are isomorphic. \\
    \texttt{test\_protocol\_case\_01}. Command in folder \texttt{test}: \\
    \texttt{../main/scheduleCompare protocol/3-cycle-abc.dot \\ protocol/3-cycle-abc.dot -v} \\
    The protocol shows: \\
    \texttt{Isomorphism 2, Graph 1, Vertex: a != 'b'; != 'c';} \\
    This protocol shows failure while constructing the second isomorphism. This is OK.
  \item Graph 1 and graph 2 have the same size and are not isomorphic due to vertex comparison with name comparison. \\
    \texttt{test\_protocol\_case\_02}. Command in folder \texttt{test}: \\
    \texttt{../main/scheduleCompare protocol/3-cycle-abc.dot \\ protocol/3-cycle-a1b1c1.dot -v} \\
    The protocol shows: \\
    \texttt{Isomorphism 1, Graph 1, Vertex: a != 'a1'; != 'b1'; != 'c1';} \\
    This shows failure while constructing the first isomorphism. Vertex a is not found in graph 2.
  \item Graph 1 and graph 2 have the same size and are not isomorphic due to vertex comparison without name comparison.
    \texttt{test\_protocol\_case\_03}. Command in folder \texttt{test}: \\
    \texttt{../main/scheduleCompare protocol/3-cycle-abc.dot \\ protocol/3-cycle-abc-par.dot -vn} \\
    The protocol shows: \\
    \texttt{Isomorphism 1, Graph 1, Vertex: a compare: -1, key: par, value1: '', value2: '1'. \\
    compare: -1, key: par, value1: '', value2: '1'. \\
    compare: -1, key: par, value1: '', value2: '1'.} \\
    The vertices have different attributes (par=1 in graph 2).
  \item Graph 1 and graph 2 have the same size and are not isomorphic due to edge comparison with name comparison.
    \texttt{test\_protocol\_case\_04}. Command in folder \texttt{test}: \\
    \texttt{../main/scheduleCompare protocol/3-cycle-abc.dot \\ protocol/broken-cycle-abc.dot -vn} \\
    The protocol shows: \\
    \texttt{Isomorphism 1, Graph 1, Vertex: a !='b'; !='c';} \\
    Mapping each vertex to the vertex with the same name fails because of the different structure.
    Construction of some other isomorphism fails due to different names.
    Graph 1 is a cycle, graph 2 has an edge from a to c. Thus, the structure is different.
  \item Graph 1 and graph 2 have the same size and are not isomorphic due to edge comparison without name comparison.
    \texttt{test\_protocol\_case\_05}. Command in folder \texttt{test}: \\
    \texttt{../main/scheduleCompare protocol/3-cycle-abc.dot \\ protocol/3-cycle-abc-xy1.dot -vn} \\
    The protocol shows: \\
    \texttt{Isomorphism 1, Graph 1, Edge: a -> b [xy]: \\ 
        Result: -1, key: type, value1: 'xy', value2: 'xy1'. \\
        Result: -1, key: type, value1: 'xy', value2: 'xy1'. \\ 
        Result: -1, key: type, value1: 'xy', value2: 'xy1'.} \\
    The edge $a \rightarrow b$ is compared to three edges of graph 2. The edge type xy is not found in graph 2.
    There is no isomorphism because graph 1 uses edge type xy and graph 2 uses edge type xy1.

  \item Graph 1 has less edges than graph 2, but the same number of vertices and graph is isomorphic to a subgraph of graph 2. \\
    \texttt{test\_protocol\_case\_06}. This is imposible, graph 2 has an edge, which has no source in graph 1.
  \item Graph 1 has less edges than graph 2, but the same number of vertices and are not isomorphic due to vertex comparison with name comparison.
    \texttt{test\_protocol\_case\_07}. Command in folder \texttt{test}: \\
    \texttt{../main/scheduleCompare protocol/3-path-a1b1c1.dot \\ protocol/3-cycle-abc.dot -vn} \\
    The protocol shows: \\
    \texttt{Isomorphism 1, Graph 1, Vertex: a1 != 'a'; != 'b'; != 'c';} \\
    Vertex a1 is not found in graph 2.
  \item Graph 1 has less edges than graph 2, but the same number of vertices and are not isomorphic due to vertex comparison without name comparison.
    \texttt{test\_protocol\_case\_08}. Command in folder \texttt{test}: \\
    \texttt{../main/scheduleCompare protocol/3-path-abc-par.dot \\ protocol/3-cycle-abc.dot -vn} \\
    The protocol shows: \\
    \texttt{Isomorphism 1, Graph 1, Vertex: a1 compare: 1, key: par, value1: '1', value2: ''. \\
    compare: 1, key: par, value1: '1', value2: ''. \\
    compare: 1, key: par, value1: '1', value2: ''.} \\
    Vertices of graph 1 have a parameter, value 1 but vertices of graph 2 have no parameter. This causes vertex comparison to fail.
  \item Graph 1 has less edges than graph 2, but the same number of vertices and are not isomorphic due to edge comparison with name comparison.
    \texttt{test\_protocol\_case\_09}. Command in folder \texttt{test}: \\
    \texttt{../main/scheduleCompare protocol/3-path-abc.dot \\ protocol/3-cycle-abc.dot -v} \\
    The protocol shows: \\
    \texttt{Isomorphism 1, Graph 1, Vertex: a != 'b'; != 'c';} \\
    There is no edge from c to a in graph 1. 

  \item Graph 1 has less edges than graph 2, but the same number of vertices and are not isomorphic due to edge comparison without name comparison.
    \texttt{test\_protocol\_case\_10}. Command in folder \texttt{test}: \\
    \texttt{../main/scheduleCompare protocol/3-path-abc.dot \\ protocol/3-cycle-abc.dot -vn} \\
    The protocol shows: nothing. \\
    There is no edge from c to a in graph 1. since vertex names are not compared, no protocol is shown.

  \item Graph 1 has less vertices than graph 2, but the same number of edges and graph 1 is isomorphic to a subgraph of graph 2. \\
    \texttt{test\_protocol\_case\_11}. 
    This is imposible. Each subset of vertices of graph 2 which has as many vertices than graph 1 has less edges than graph 1.
    Thus there is no isomorphism.
    Dummy test case.
  \item Graph 1 has less vertices than graph 2, but the same number of edges and are not isomorphic due to vertex comparison with name comparison.
    \texttt{test\_protocol\_case\_12}. Command in folder \texttt{test}: \\
    \texttt{../main/scheduleCompare protocol/3-cycle-abc.dot \\ protocol/4-path-a1b1c1d1.dot -v} \\
    The protocol shows: \\
    \texttt{Isomorphism 1, Graph 1, Vertex: a !='a1'; !='b1'; !='c1'; \\ !='d1';} \\
    Vertex a of graph 1 is not found in graph 2.
    Graph 1 is a cycle with 3 vertices and graph 2 is a path with 4 vertices.
  \item Graph 1 has less vertices than graph 2, but the same number of edges and are not isomorphic due to vertex comparison without name comparison.
    \texttt{test\_protocol\_case\_13}. Command in folder \texttt{test}: \\
    \texttt{../main/scheduleCompare protocol/3-cycle-abc.dot \\ protocol/4-path-abcd-par.dot -vn} \\
    The protocol shows: \\
    \texttt{Isomorphism 1, Graph 1, Vertex: a compare: -1, key: par, value1: '', value2: '1'. \\
    compare: -1, key: par, value1: '', value2: '1'. \\
    compare: -1, key: par, value1: '', value2: '1'.} \\
    Graph 1 is a cycle with 3 vertices and graph 2 is a path with 4 vertices.
    Vertices in graph 2 have a parameter value 1. Vertices in graph 1 have no parameter.
  \item Graph 1 has less vertices than graph 2, but the same number of edges and are not isomorphic due to edge comparison with name comparison.
    \texttt{test\_protocol\_case\_14}. Command in folder \texttt{test}: \\
    \texttt{../main/scheduleCompare protocol/3-cycle-abc.dot \\ protocol/4-path-abcd.dot -v} \\
    The protocol shows: \\
    \texttt{Isomorphism 1, Graph 1, Vertex: a != 'b'; != 'c'; != 'd';} \\
    Graph 1 is a cycle with 3 vertices and graph 2 is a path with 4 vertices.
  \item Graph 1 has less vertices than graph 2, but the same number of edges and are not isomorphic due to edge comparison without name comparison.
    \texttt{test\_protocol\_case\_15}. Command in folder \texttt{test}: \\
    \texttt{../main/scheduleCompare protocol/3-cycle-abc.dot \\ protocol/4-path-abcd.dot -vn} \\
    The protocol shows: The protocol shows: nothing (protocol implementation to be improved).\\
    Graph 1 is a cycle with 3 vertices and graph 2 is a path with 4 vertices.

  \item Graph 1 has less vertices than graph 2 and less edges and graph 1 is isomorphic to a subgraph of graph 2.
    \texttt{test\_protocol\_case\_16}. Command in folder \texttt{test}: \\
    \texttt{../main/scheduleCompare protocol/3-cycle-abc.dot \\ protocol/3cycles-abcde.dot -v} \\
    The protocol shows: \\
    \texttt{Isomorphism 2, Graph 1, Vertex: a !='b'; !='c'; !='d'; \\ !='e';
    Isomorphism 2, Graph 1, Vertex: c != 'd';} \\
    This protocol shows that the construction of the second isomorphism fails.
    There is one isomorphism (0, 0) (1, 1) (2, 2). 
  \item Graph 1 has less vertices than graph 2 and less edges and are not isomorphic due to vertex comparison with name comparison. \\
    \texttt{test\_protocol\_case\_17}. Command in folder \texttt{test}: \\
    \texttt{../main/scheduleCompare protocol/3-cycle-abc.dot \\ protocol/3cycles-a1b1c1d1e1.dot -v} \\
    The protocol shows: \\
    \texttt{Isomorphism 1, Graph 1, Vertex: a != 'a1'; != 'b1'; \\ != 'c1'; != 'd1'; != 'e1';} \\
    Vertex a is not found in graph 2.
  \item Graph 1 has less vertices than graph 2 and less edges and are not isomorphic due to vertex comparison without name comparison. \\
    \texttt{test\_protocol\_case\_18}. Command in folder \texttt{test}: \\
    \texttt{../main/scheduleCompare protocol/3-cycle-abc.dot \\ protocol/3cycles-abcde-par.dot -vn} \\
    The protocol shows: \\
    \texttt{Isomorphism 1, Graph 1, Vertex: a compare: -1, key: par, value1: '', value2: '1'. \\
    compare: -1, key: par, value1: '', value2: '1'. \\
    compare: -1, key: par, value1: '', value2: '1'. \\
    compare: -1, key: par, value1: '', value2: '1'. \\
    compare: -1, key: par, value1: '', value2: '1'.} \\
    Vertices in graph 2 have a parameter value 1. Vertices in graph 1 have no parameter.
  \item Graph 1 has less vertices than graph 2 and less edges and are not isomorphic due to edge comparison with name comparison. \\
    \texttt{test\_protocol\_case\_19}. Command in folder \texttt{test}: \\
    \texttt{../main/scheduleCompare protocol/3-cycle-abc.dot \\ protocol/5-path-abcde.dot -v} \\
    The protocol shows: \\
    \texttt{Isomorphism 1, Graph 1, Vertex: a!='b'; !='c'; !='d'; \\ !='e';} \\
    Graph 1 is a cycle with 3 vertices and graph 2 is a path with 5 vertices.
  \item Graph 1 has less vertices than graph 2 and less edges and are not isomorphic due to edge comparison without name comparison. \\
    \texttt{test\_protocol\_case\_20}. Command in folder \texttt{test}: \\
    \texttt{../main/scheduleCompare protocol/3-cycle-abc.dot \\ protocol/5-path-abcde.dot -vn} \\
    The protocol shows: nothing (protocol implementation to be improved). \\
    Graph 1 is a cycle with 3 vertices and graph 2 is a path with 5 vertices.
\end{enumerate}

\chapter{Source and Branches}
The source code is in branches \texttt{dm-analysis} and \texttt{dm-analysis-2},
folder \texttt{modules/ftm/analysis/sourceCompare/}. The subfolders
\begin{enumerate}
  \item doc for the documentation,
  \item main for the source files,
  \item test for the test framework and the test input
\end{enumerate}
\end{document}
