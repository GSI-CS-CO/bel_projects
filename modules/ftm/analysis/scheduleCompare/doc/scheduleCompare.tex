\documentclass[12pt,a4paper]{report}
\usepackage{tabularx, multirow}
\begin{document}
\begin{titlepage}
\begin{center}
\vspace{2em}
\Huge{ScheduleCompare}\\[2cm]
\Large{Documentation}\\[2cm]
\begin{large}
\begin{tabularx}{\textwidth}{Xl}
Version & 1.0\\
Last updated & 20201-01-22\\
\vspace{1.5cm}\\
Author & Martin Skorsky\\
Department & ACO \\
Group & TOS\\
Contact & Martin Skorsky
\end{tabularx}%
\end{large}

\vfill

\end{center}
\end{titlepage}
\tableofcontents
\chapter{Purpose of ScheduleCompare}
The tool \texttt{scheduleCompare} tests if two graphs describe the same schedule. The graphs are written as dot-files.

The output is either "graphs represent the same schedule" or "graphs represent different schedules and the differences are \textellipsis".
The output in the later case should show at least the first vertex with non-matching attributes.

The algorithm used ist the method \texttt{vf2\_subgraph\_iso} from the Boost Graph Library. This tests if a graph $g_1$ is isomorph 
to a subgraph og graph$g_2$. Thus, we have to test that $g_1$ is a subgraph of $g_2$ and $g_2$ is a subgraph $g_1$ to check 
that the two graphs are isomorph. This test can be shortened by testing that $g_1$ is a subgraph of $g_2$ and the number 
of vertices are equal $|V(g_1)| = |V(g_2)|$ and the number of edges are equal $|E(g_1)| = |E(g_2)|$.

May be, it is also interesting to know that $g_1$ is a subgraph of $g_2$. In the following table, $g_1 S g_2$ 
means $g_1$ is possibly a subgraph of $g_2$. In addition, $g_1 S g_2$ means $g_1$ is possibly isomorphic to $g_2$. 
\begin{table}
\label{tab:cardinalities}
\begin{tabular}[t]{|c|ccc|}
\hline
                     & $|E(g_1)|<|E(g_2)|$ & $|E(g_1)|=|E(g_2)|$ & $|E(g_1)|>|E(g_2)|$ \\ \hline
 $|V(g_1)|<|V(g_2)|$ &          S          &          S          &          -  \\
 $|V(g_1)|=|V(g_2)|$ &          S          &          I          &          -  \\
 $|V(g_1)|>|V(g_2)|$ &          -          &          -          &          -  \\ \hline
\end{tabular}
\end{table}

TODO: document prerequisites, like dot files syntactical correct, not a null graph.
\chapter{Arguments and Options}
Usage: \texttt{scheduleCompare $<$dot file 1$>$ $<$dot file 2$>$}

Options:
\begin{enumerate}
	\item -h: help and usage.
	\item -s: silent, no output, only return code. Usefull for automated tests.
	\item -v: verbose, output of input graphs as dot text.
	\item -vv: super verbose, in addition to verbose more output.
\end{enumerate}

Return Codes:
\begin{enumerate}
	\item 0 EXIT_SUCCESS, graphs are isomorphic.
	\item 1 NOT_ISOMORPHIC, graphs are not isomorphic.
	\item 2 SUBGRAPH_ISOMORPHIC, graph is isomorphic to a subgraph of the larger graph.
	\item 11 BAD_ARGUMENTS, unknown arguments on command line.
	\item 12 MISSING_ARGUMENT, at least one of the file names is missing.
	\item 13 FILE_NOT_FOUND, one of the dot files not found.
	\item 14 USAGE_MESSAGE, usage message (help) displayed.
	\item negative values are UNIX signals
\end{enumerate}

\chapter{Testframework and Test Cases}
First attempt for the test framework are unit tests with Python. A test case in general calls \texttt{scheduleCompare} with two dot files and examines the output. For this, the return code of \texttt{scheduleCompare}.

\begin{enumerate}
	\item 9 test cases for the cardinalities from table~\ref{tab:cardinalities}
	\item 1815 dot files (11 * 11 *15) from \texttt{singleEdgeTest}. Each two of these are compared. Only if a dot file is compared with itself, the graphs should be isomorphic. These are 3.294.225 combinations.
	\item same graph, but different names for the vertices.
	\item same graph, but different types for vertices with the same name.
	\item same graph, but different types for edges between the same vertices. Vertex names and types are the same.
\end{enumerate}

\chapter{Source and Branches}
The source code is in branch \texttt{dm-analysis}, \\ 
folder \texttt{modules/ftm/analysis/sourceCompare/}. The subfolders
\begin{enumerate}
	\item doc for the documentation,
	\item main for the source files,
	\item test for the test framework and the test input
\end{enumerate}
\end{document}
