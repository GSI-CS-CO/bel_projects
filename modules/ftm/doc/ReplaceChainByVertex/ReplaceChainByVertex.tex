\documentclass[12pt,a4paper]{report}
% Language: English
\pdfminorversion=7
\usepackage[pdftex]{graphicx}
\usepackage{changepage}
\usepackage{xcolor}
\usepackage{listings}

\lstdefinestyle{dotfiles}{
  escapeinside={(*@}{@*)}, % (*@\label{mylabel}@*)
  numbers=left,
  stepnumber=1,
  numberstyle=\tiny,
  numbersep=10pt,
  captionpos=b,
  belowcaptionskip=1\baselineskip,
  breaklines=true,
  keepspaces=true,
  columns=flexible,
  language=C,
  showstringspaces=false,
  basicstyle=\scriptsize\ttfamily,
  keywordstyle=\color{green!40!black},
  commentstyle=\itshape\color{purple!40!black},
  identifierstyle=\color{blue},
  stringstyle=\color{red},
  tabsize=2,
  morekeywords={digraph, graph, subgraph, edge, node, color, style, shape, fillcolor},
}

\newcommand{\ry}{\rotatebox{90}}
\begin{document}

\begin{titlepage}
\vspace{2cm}
\begin{center}
\Huge{In a Schedule Replace a Chain by a Vertex}

\Large{Martin Skorsky}

\Large{Last change: 2023-01-24}
\end{center}
\vfill
\end{titlepage}

\tableofcontents

\section*{Introduction}
In a schedule graph we have often long chains which describe a series of timing events.
When looking at the drawing of the graph, there are a lot of vertices in chains and
these hide the structure of the schedule. This structure is determined by those
vertices with more than one incoming edge or more than one outgoing edge.

To get more readable drawings we want to replace each chain by a new vertex which is
connected to the rest of the graph in the same way a the chain. This should reduce
the number of vertices significantly. The number of edges is in the same range as
the number of vertices. For a current schedule from the integration environment
the number of vertices decreases from about 1,000 vertices to less than 200 vertices.
The number of edges decreases from 1,200 edges to about 300 edges.

This documentation describes an algorithm to replace in a schedule graph, which is a
directed graph, a chain by a new vertex.

\chapter{Replace a Chain by a Vertex}
The algorithm operates on schedule graphs and other directed graphs which are described
with the dot language.
The algorithm is implemented in C++ with the Boost Graph Library (Version 1.69 or higher).

For the description we need some definitions. Assume, we have a directed graph $G = (V,E)$ where
$V$ is the set of vertices an $E$ is the set of edges, a subset of $V \times V$. Each
edge is a pair $(v_1, v_2)$ of two vertices. $v_1$ is the predecessor $p(v_2)$ of $v_2$. $v_2$ is
the successor $s(v_1)$ of $v_1$. $Out(v_1)$ is the set of outgoing edges of $v_1$ and $In(v_1)$
is the set of incoming edges of $v_1$. $|Out(v_1)|$ is the out-degree of $v_1$ and $|In(v_1)|$
is the in-degree of $v_1$.

A subgraph $S = (U,F)$ of $G$ is a graph where $U \subset V$ and $F \subset E$ and $u_1, u_2 \in U$
with $(u_1, u_2) \in E$ implies $(u_1, u_2) \in F$. In words: an edge of $G$ where both vertices
are in the subgraph $S$ is also an edge of the subgraph.

In the following $In(v)$ and $Out(v)$ are with repect to $G$ and not the chain $C$.

\begin{enumerate}
\item Chain: A chain is a connected subgraph $C$ of $G$ and all vertices $v\in V(C)$ have
$|In(v)| \le 1$ and $|Out(v)| \le 1$. Remark 1: since the subgraph is connected, there is at most
one vertex with $|In(v)| = 0$ or $|Out(v) = 0$. Remark 2: with the algorithm described below we
search for maximal chains in $G$. Remark 3: Cycles are also chains.

\item Begin of chain: a vertex $b$ of the chain, which has no predecessor in the chain.
Remark: it is $|In(b)| \le 1$. In the case $|In(b)| = 1$ there is a unique vertex
$a \in V(G) \setminus V(C)$ which is the predecessor of $b$.

\item End of chain: a vertex $c$ of the chain, which has no successor in the chain.
Remark: it is $|Out(c)| \le 1$. In the case $|Out(c)| = 1$ there is a unique vertex
$d \in V(G) \setminus V(C)$ which is the successor of $c$.

\end{enumerate}

\chapter{Requirements}
The algorithm is designed with the following requirements:

\begin{enumerate}
\item A maximal chain is replaced by a new vertex. This new vertex is connected to the
predecessor of the begin of the chain and the successor of the end of the chain if these exist.

\item There is one condition that decides if a vertex is deleted as part of a chain.
There is no list of candidates for deletion which is revised later.

\item The label of a new vertex is the concatenation of the labels in the chain with some
modifications. If the chain is two or three vertices long, the new label is concatenated
from these vertices from begin of chain to end of chain. If the chain is longer than 3
vertices, the new label is concatenated from the begin of chain, three dots, the end of chain.
The same is done for the names of the vertices. If the labels are empty, use the names for
the labels.

\item The properties of the new vertex are the properties of the begin of chain. This
includes pattern, fillcolor, shape, and type.

\item The properties of the new edges are the properties of the corresponding edges
of $G$. This includes color and type.

\item If the graph $G$ has properties for positioning (including pos, \_draw\_, \_hdraw\_, \_ldraw\_)
these properties need a special handling. The position of a new vertex should be the average
position of the vertices in the chain. The positions of edges need more work. This has low priority.
For readable drawings of the new graph it is better to rely on the wellknown positioning
algorithms of graphviz and use a new rendering of the new graph. Unfortunately, new vertices
may be placed differently than the underlying chain in the new graph drawing.

\end{enumerate}

\chapter{Algorithm}
This algorithm replaces a maximal chain with a new vertex with two edges. To replaces all chains,
run this algorithm until no more chains are detected.
\section{Check that a vertex is replaced}
\subsection{Find start of chain}
For a vertex $v$ we use the conditions $|In(v)| = 0$, $|In(v)| = 1$, $|In(v)| > 1$,
$|Out(v)| = 0$, $|Out(v)| = 1$, $|Out(v)| > 1$. Thus we have nine cases to check.

'go to next vertex' means test next vertex in the numbering of the vertices.

For all vertices $v \in V(G)$ do
\begin{enumerate}
\item If $|In(v)| = 0$ and $|Out(v)| = 0$ then $v$ is an isolated vertex. Go to next vertex.
\item If $|In(v)| = 1$ and $|Out(v)| = 0$ then $v$ is the end of a chain. If $p(v)$ in chain, go to $p(v)$. Otherwise go to next vertex.
\item If $|In(v)| > 1$ and $|Out(v)| = 0$ then $v$ is not in a chain. Go to next vertex.
\item If $|In(v)| = 0$ and $|Out(v)| = 1$ then $v$ is the start of a chain. Break loop and go to next step.
\item If $|In(v)| = 1$ and $|Out(v)| = 1$ then $v$ is in a chain. If $p(v)$ in chain, go to $p(v)$. Otherwise $v$ is start of chain. Break loop and go to next step.
\item If $|In(v)| > 1$ and $|Out(v)| = 1$ then $v$ is not in a chain. Go to next vertex.
\item If $|In(v)| = 0$ and $|Out(v)| > 1$ then $v$ is not in a chain. Go to next vertex.
\item If $|In(v)| = 1$ and $|Out(v)| > 1$ then $v$ is not in a chain. Go to next vertex.
\item If $|In(v)| > 1$ and $|Out(v)| > 1$ then $v$ is not in a chain. Go to next vertex.
\end{enumerate}
Simplyfing the cases yields:

For all vertices $e \in V(G)$ do
\begin{enumerate}
\item If ($|In(v)| = 0$ and $|Out(v)| = 0$) or $|In(v)| > 1$ or $|Out(v)| > 1$ go to next vertex.
\item If $|In(v)| = 1$ and $|Out(v)| = 0$ then $v$ is the end of a chain. If $p(v)$ in chain, go to $p(v)$. Otherwise go to next vertex.
\item If $|In(v)| = 0$ and $|Out(v)| = 1$ then $v$ is the start of a chain. Break loop and go to next step.
\item If $|In(v)| = 1$ and $|Out(v)| = 1$ then $v$ is in a chain. If $p(v)$ in chain, go to $p(v)$. Otherwise $v$ is start of chain. Break loop and go to next step.
\end{enumerate}

\subsection{Test that a vertex can be replaced}
Vertex $v$ is start of chain ($|In(v)| \le 1$ and $|Out(v)| = 1$). Since the successor
$s = s(v)$ is unique, test $s$. If $|In(s)| = 1$ and $|Out(s)| \le 1$ then $s$
is in this chain and $v$ can be deleted. Do the actions for $v$ and apply
this test to $s$.

\subsection{Actions}
Add vertex $v$ to the set of vertices to delete.

If $v$ is start of chain, create a new vertex $n$ and copy the properties from $v$.
If a predecessor $p = p(v)$ exists, create a new edge $(p,n)$ and copy
the properties from $(p,v)$.

If $v$ is the end of chain and a successor $s = s(v)$ exists, create a
new edge $(n,s)$ and copy the properties from $(v,s)$.

\section{Replace a Chain by a Vertex with two edges}
The order of the operations is important since deletion of a vertex changes
the numbering of the vertex set and this may destroy the structure of the graph.

\begin{enumerate}
\item Add the new vertex $n$ to $G$.
\item Add the new edges $(p,n)$ and $(n,s)$.
\item For all vertices in the chain delete the edges into and out of this vertex.
\item For all vertices in the chain delete the vertex itself. This has to be done
in reverse order of the numbering.
\end{enumerate}

\chapter{Examples}
The original schedule in the integration environment is shown in three parts in~\ref{fig:tsl020-1},
\ref{fig:tsl020-2}, \ref{fig:tsl020-3}.
    \begin{figure}
        \centering
        \includegraphics*[width=1.0\textwidth,keepaspectratio]{tsl020-1.pdf}
        \caption{Schedule from integration environment, part 1 (top)}
        \label{fig:tsl020-1}
    \end{figure}
    \begin{figure}
        \centering
        \includegraphics*[width=1.0\textwidth,keepaspectratio]{tsl020-2.pdf}
        \caption{Schedule from integration environment, part 2 (middle)}
        \label{fig:tsl020-2}
    \end{figure}
    \begin{figure}
        \centering
        \includegraphics*[width=1.0\textwidth,keepaspectratio]{tsl020-3.pdf}
        \caption{Schedule from integration environment, part 3 (bottom)}
        \label{fig:tsl020-3}
    \end{figure}
The compacted schedule is shown in~\ref{fig:tsl020-compact}.
    \begin{figure}
        \centering
        \includegraphics*[height=0.95\textheight,keepaspectratio]{tsl020-compact.pdf}
        \caption{Schedule from integration environment, chains replaced}
        \label{fig:tsl020-compact}
    \end{figure}
These figures give a brief idea of the achievements of this method to
replace chains by vertices. The analysis of the structure can be done
with tools like \texttt{xdot}, where you can zoom into the graph. The original
schedule has $1057$ vertices and $1173$ edges. The compacted schedule
has $177$ vertices and $293$ edges.

\chapter{Tests}
    \begin{figure}
        \centering
        \includegraphics*[width=0.45\textwidth,keepaspectratio]{chain1.pdf}
        \caption{Schedule with one-element chain. Nothing to replace.}
        \label{fig:chain1}
    \end{figure}
    \begin{figure}
        \centering
        \includegraphics*[width=0.45\textwidth,keepaspectratio]{chain3.pdf}
        \caption{Schedule with one three-element chain. The chain is replaced by one vertex.}
        \label{fig:chain3}
    \end{figure}
    \begin{figure}
        \centering
        \includegraphics*[width=0.45\textwidth,keepaspectratio]{chain3-compact.pdf}
        \caption{Schedule with one three-element chain compacted.}
        \label{fig:chain3-compact}
    \end{figure}
    \begin{figure}
        \centering
        \includegraphics*[width=0.45\textwidth,keepaspectratio]{chain2x3.pdf}
        \caption{Schedule with two three-element chains. Both chains are replaced by one vertex.}
        \label{fig:chain2x3}
    \end{figure}
    \begin{figure}
        \centering
        \includegraphics*[width=0.45\textwidth,keepaspectratio]{chain2x3-compact.pdf}
        \caption{Schedule with two three-element chains compacted.}
        \label{fig:chain2x3-compact}
    \end{figure}
    \begin{figure}
        \centering
        \includegraphics*[width=0.45\textwidth,keepaspectratio]{cycle3.pdf}
        \caption{Schedule with one three-element cycle. The cycle is replaced by a two-element cycle.}
        \label{fig:cycle3}
    \end{figure}
    \begin{figure}
        \centering
        \includegraphics*[width=0.45\textwidth,keepaspectratio]{cycle3-compact.pdf}
        \caption{Schedule with one three-element cycle compacted.}
        \label{fig:cycle3-compact}
    \end{figure}
    \begin{figure}
        \centering
        \includegraphics*[width=0.45\textwidth,keepaspectratio]{cycle2x3.pdf}
        \caption{Schedule with two three-element cycles. Both cycles are replaced by one vertex.}
        \label{fig:cycle2x3}
    \end{figure}
    \begin{figure}
        \centering
        \includegraphics*[width=0.45\textwidth,keepaspectratio]{cycle2x3-compact.pdf}
        \caption{Schedule with two three-element cycles compacted.}
        \label{fig:cycle2x3-compact}
    \end{figure}
    \begin{figure}
        \centering
        \includegraphics*[width=0.45\textwidth,keepaspectratio]{parallel1.pdf}
        \caption{Schedule with a parallel chain with one element. Nothing to replace}
        \label{fig:parallel1}
    \end{figure}
    \begin{figure}
        \centering
        \includegraphics*[width=0.45\textwidth,keepaspectratio]{parallel1x2.pdf}
        \caption{Schedule with a parallel chain with two elements.}
        \label{fig:parallel1x2}
    \end{figure}
    \begin{figure}
        \centering
        \includegraphics*[width=0.45\textwidth,keepaspectratio]{parallel1x2-compact.pdf}
        \caption{Schedule with a parallel chain with two elements compacted.}
        \label{fig:parallel1x2-compact}
    \end{figure}
    \begin{figure}
        \centering
        \includegraphics*[width=0.45\textwidth,keepaspectratio]{parallel2.pdf}
        \caption{Schedule with two one-element chains. Nothing to replace.}
        \label{fig:parallel2}
    \end{figure}
    \begin{figure}
        \centering
        \includegraphics*[width=0.45\textwidth,keepaspectratio]{parallel2x2.pdf}
        \caption{Schedule with two parallel chains with two elements.}
        \label{fig:parallel2x2}
    \end{figure}
    \begin{figure}
        \centering
        \includegraphics*[width=0.45\textwidth,keepaspectratio]{parallel2x2-compact.pdf}
        \caption{Schedule with two parallel chains with two elements compacted.}
        \label{fig:parallel2x2-compact}
    \end{figure}
    \begin{figure}
        \centering
        \includegraphics*[width=0.45\textwidth,keepaspectratio]{star4.pdf}
        \caption{Schedule with a four-element star. Nothing to replace.}
        \label{fig:star4}
    \end{figure}

\end{document}
