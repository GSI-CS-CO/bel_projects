\newglossaryentry{sched:type}{
name={type},
description={Determines the type of this node (e.g. block, tmsg, etc.)},
type=sched}

\newglossaryentry{sched:type:tmsg}{
name={tmsg},
description={Timing Message. Broadcasts a message to all timing receivers},type=sched,
parent={sched:type}
}

\newglossaryentry{sched:type:block}{
name={block},
description={Fixed length block. Terminates a sequence of nodes and defines the sequence's length in time. See \gls{sched:tperiod}.},
type=sched,
parent={sched:type}
}

\newglossaryentry{sched:type:blockalign}{
name={blockalign},
description={Alignment block. Extends its own length so the time sum will become a multiple of the time grid (currently \SI{10}{\micro\second}).
See \gls{sched:tperiod}, \gls{sched:type:block}.},
type=sched,
parent={sched:type}
}

\newglossaryentry{sched:type:flow}{
name={flow},
description={Causes one or more repetitions of a flow command to be written to the target block's queue, changing it's default successor (e.g. branch, loop, etc).},
type=sched,
parent={sched:type}
}

\newglossaryentry{sched:type:flush}{
name={flush},
description={Causes a flush command to be written to the target block's queue, emptying any selected queues. Can optionally also override the default successor.},
type=sched,
parent={sched:type}
}

\newglossaryentry{sched:type:wait}{
name={wait},
description={Causes a wait command to be written to the target block's queue, temporarily stretching it's length in time.},
type=sched,
parent={sched:type}
}

\newglossaryentry{sched:type:noop}{
name={nop},
description={Causes one or more No Operation command(s) to be written to the target block's queue.
No direct effect, but can together with flow create a sequence of default and alternative successors.},
type=sched,
parent={sched:type}
}

\newglossaryentry{sched:type:origin}{
name={origin},
description={Assigns a sequence of nodes to a thread. The node pointed to by an edge of type \gls{edge:origindst} is the first which is assigned to the thread.},
type=sched,
parent={sched:type}
}

\newglossaryentry{sched:type:startthread}{
name={startthread},
description={Starts the thread which is given by the thread attribute. The attribute \gls{sched:startoffs} is the offset between block start and thread start.},
type=sched,
parent={sched:type}
}



\newglossaryentry{sched:node_id}{
name={node\_id},
description={Tag of the node's name. Internal use only},
type=sched}

\newglossaryentry{sched:cpu}{
name={cpu},
description={Index of the CPU core this node will reside in},
type=sched}

\newglossaryentry{sched:thread}{
name={thread},
description={Index of the thread core which handles this node.
To be replaced by auto-balancer algorithm. The thread number from 0 to 7 to which an origin node assigns a sequence of nodes. A startthread node starts this thread.},
type=sched}

\newglossaryentry{sched:flags}{
name={flags},
description={Aggregation of certain node flags, internal use only},
type=sched}

\newglossaryentry{sched:patentry}{
name={patentry},
description={If true, this node is an entry point to its pattern},
type=sched}

\newglossaryentry{sched:patexit}{
name={patexit},
description={If true, this node is an exit point from its pattern},
type=sched}

\newglossaryentry{sched:pattern}{
name={pattern},
description={The name of this node's pattern},
type=sched}

\newglossaryentry{sched:bpentry}{
name={bpentry},
description={If true, this node is an entry point to its beamprocess. Currently not supported},
type=sched}

\newglossaryentry{sched:bpexit}{
name={bpexit},
description={If true, this node is an exit point from the beamprocess it belongs to. Currently not supported},
type=sched}

\newglossaryentry{sched:beamproc}{
name={beamproc},
description={The name of the beamprocess this node belongs to. Currently not supported},
type=sched}

\newglossaryentry{sched:permanent}{
name={permanent},
description={If true, the corresponding flow command will permanently change the default destination of the target block, else the change lasts only during the execution of the command},
type=sched}

\newglossaryentry{sched:tperiod}{
name={tperiod},
description={The duration of a block. Amount of time in \SI{}{\nano\second} it adds to the current time sum when processed},
type=sched}

\newglossaryentry{sched:qil}{
name={qil},
description={If true, adds a command queue of top priority (Interlock) to this block},
type=sched}

\newglossaryentry{sched:qhi}{
name={qhi},
description={If true, adds a command queue of medium priority (High) to this block},
type=sched}

\newglossaryentry{sched:qlo}{
name={qlo},
description={If true, adds a command queue of low priority (Low) to this block},
type=sched}

\newglossaryentry{sched:toffs}{
name={toffs},
description={The time offset \SI{}{\nano\second} relative to the current timesum at which this node is executed. Negative offsets can be used for debugging to force late events},
type=sched}

\newglossaryentry{sched:id}{
name={id},
description={Addresses the 64b ID field of a timing event. Sub-Id fields can be used instead, see \gls{id:id}},
type=sched}

\newglossaryentry{sched:tvalid}{
name={tvalid},
description={Time in \SI{}{\nano\second} at/after which this command will be valid (executable), can be absolute or relative depending on \gls{sched:vabs} flag},
type=sched}

\newglossaryentry{sched:vabs}{
name={vabs},
description={Chooses whether tvalid is interpreted as absolute value or an offset to current time sum. When using this as a loop initialiser head, \emph{always} use vabs=true and tvalid=0!},
type=sched}

\newglossaryentry{sched:prio}{
name={prio},
description={Chooses the command queue level this command will be written to at the target block. (0 (lo), 1(mid), 2(hi) )},
type=sched}

\newglossaryentry{sched:qty}{
name={qty},
description={Number of times this command will be repeated. The generated command will be executed until qty reaches 0, then it will be popped from the block's queue},
type=sched}

\newglossaryentry{sched:twait}{
name={twait},
description={Timespan in \SI{}{\nano\second} a wait command is goid to wait. Can absolute or relative to current time, depending on flag},
type=sched}

\newglossaryentry{sched:par}{
name={par},
description={Transparent parameter field, part of a timing message. Is passed through to a listening ECA channel in timing receiver},
type=sched}

\newglossaryentry{sched:tef}{
name={tef},
description={Fractional (sub-nanosecond) time extension field, part of a timing message. Currently not interpreted by ECA)},
type=sched}

\newglossaryentry{sched:startoffs}{
name={startoffs},
description={The start offset in \SI{}{\nano\second} for a startthread node.},
type=sched}

\newglossaryentry{edge:defdst}{
name={defdst},
description={Points to default successor node},
type=sched}

\newglossaryentry{edge:altdst}{
name={altdst},
description={Points to an alternative successor node},
type=sched}

\newglossaryentry{edge:target}{
name={target},
description={Points to target block receiving a command},
type=sched}

\newglossaryentry{edge:flowdst}{
name={flowdst},
description={Points to destination node of a flow command},
type=sched}

\newglossaryentry{edge:flushovr}{
name={flushovr},
description={Points to destination override node of a flush command},
type=sched}

\newglossaryentry{edge:switchdst}{
name={switchdst},
description={Points to destination node of a switch statement},
type=sched}

\newglossaryentry{edge:origindst}{
name={origindst},
description={Points to destination node of an origin node},
type=sched}
