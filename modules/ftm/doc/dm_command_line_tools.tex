\subsection{Tool dm-cmd}
\label{Tool_dm-cmd}
\begin{lstlisting}[style = helptext]
dm-cmd v0.36.3, build date 20220602
Sends a command or dotfile of commands to the DM
There are global, local and queued commands

Usage: dm-cmd [OPTION] <etherbone-device> <command>
                              [target node] [parameter]


General Options:
  -c <cpu-idx>              Select CPU core by index, default is 0
  -t <thread-idx>           Select thread inside selected CPU core
                            by index, default is 0
  -v                        Verbose operation, print more details
  -d                        Debug operation, print everything
  -i <command .dot file>    Run commands from dot file
  status                    Show status of all threads and cores
                            (default)
  details                   Show time statistics and detailed
                            information on uptime and recent changes
  clearstats                Clear all status and statistics info
  gathertime <Time / ns>    Set msg gathering time for priority queue
  maxmsg <Message Quantity> Set maximum messages in a packet for
                            priority queue
  running                   Show bitfield of all running threads
                            on this CPU core
  heap                      Show current scheduler heap
  startpattern <pattern>    Request start of selected pattern
  abortpattern <pattern>    Try to immediately abort selected pattern
  chkrem       <pattern>    Check if removal of selected pattern
                            would be safe
  starttime <Time / ns>     Set start time for this thread
  preptime <Time / ns>      Set preparation time (lead) for this thread
  deadline                  Show next deadline for this thread
  origin <target>           Set the node with which selected
                            thread will start
  origin                    Return the node with which selected
                            thread will start
  cursor                    Show name of currently active node
                            of selected thread
  hex <target>              Show hex dump of selected Node
  queue <target>            Show content of all queues
\end{lstlisting}
\texttt{target} is a block node where the queues are configured.
\begin{lstlisting}[style = helptext]
  rawqueue <target>         Dump all meta information of the command
                            queues of the block <target> including
                            commands
  start                     Request start of selected thread.
                            Requires a valid origin.
  stop                      Request stop of selected thread.
                            Does reverse lookup of current pattern,
                            prone to race condition
  abort                     Immediately abort selected thread.
  halt                      Immediately aborts all threads on all CPUs.
  lock <target>             Locks all queues of a block for asynchronous
                            queue manipulation mode. This makes the
                            queues invisible to the DM and allowing
                            modification during active runtime.
                            ACTIVE LOCK MEANS DM WILL NEITHER WRITE TO
                            NOR READ FROM THIS BLOCK'S QUEUES!
  clear <target>            Clears all queues of a locked block
                            allowing modification/refill during
                            active runtime.
  unlock <target>           Unlocks all queues of a block, making
                            them visible to the DM.
  asyncflush <target>  <prios>         Flushes all pending commands
                            of given priorities (3b Hi-Md-Lo ->
                            0x0..0x7) in an locked block of the schedule
  unlock <target>                      Unlocks a block from
                            asynchronous queue manipulation mode
  showlocks                            Lists all currently locked blocks
  staticflush <target> <prios>         Flushes all pending commands
                            of given priorities (3b Hi-Md-Lo ->
                            0x0..0x7) in an inactive (static)
                            block of the schedule
  staticflushpattern <pattern> <prios> Flushes all pending commands
                            of given priorities (3b Hi-Md-Lo ->
                            0x0..0x7) in an inactive (static)
                            pattern of the schedule
  rawvisited [<target>]     Show 1 for a visited node, 0 for not visited.
                            If no target node is given, show all nodes.

Queued commands (viable options in square brackets):
  stop <target> [laps]                 Request stop at
                                       selected block (flow to idle)
  stoppattern  <pattern> [laps]        Request stop of selected pattern
  noop <target> [lapq]                 Placeholder to stall succeeding
                                       commands, has no effect itself
  flow <target> <destination node>     [lapqs]  Changes schedule
                                       flow to <Destination Node>
  flowpattern <target pat.> <dst pat.> [lapqs]  Changes schedule
                                       flow to <Destination Pattern>
  relwait <target> <wait time / ns> [laps] Changes Block period
                                       to <wait time>
  abswait <target> <wait time / ns> [lap] Stretches Block
                                       period until <wait time>
  flush <target> <prios> [lap]         Flushes all pending
                                       commands of given priorities
                                       (3b, 0x0..0x7) le cmd priority
Options for queued commands:
  -l <Time / ns>           Time in ns after which the command will
                           become active, default is 0 (immediately)
  -a                       Interprete valid time of command as
                           absolute. Default is relative (current
                           WR time is added)
  -p <priority>            The priority of the command (0 = Low,
                           1 = High, 2 = Interlock), default is 0
  -q <quantity>            The number of times the command will
                           be inserted into the target queue,
                           default is 1
  -s                       Changes to the schedule are permanent

Diagnostics:
  diag                               Show time statistics and
                                     detailed information on
                                     uptime and recent changes
  cleardiag                          Clears all CPU and HW statistics
                                     and details
  cfghwdiag <TAI / ns> <Stall / ns>  Sets observation window for
                                     ECA TAI time continuity and
                                     CPU stall streaks
  starthwdiag                        Starts HW diagnostic data
                                     acquisition
  stophwdiag                         Stops HW diagnostic data
                                     acquisition
  clearhwdiag                        (Todo)
  cfgcpudiag <Warn. Threshold / ns>  Globally sets warning
                                     threshold for minimum message
                                     dispatch lead
  clearcpudiag                       Clears CPU statistics for
                                     given index
\end{lstlisting}

\subsection{Tool dm-sched}
\label{Tool_dm-sched}
\begin{lstlisting}[style = helptext]
dm-sched v0.36.3, build date 20220602
Creates Binary Data for the DataMaster (DM) from Schedule Graphs
(.dot files) and uploads/downloads to/from CPU Core <m> of the
DM (CPU currently specified in schedule as cpu=<m>).

Usage: dm-sched <etherbone-device> <Command> <.dot file>

Commands:
  status                 Gets current DM schedule state (default)
  dump                   Gets current DM schedule
  clear                  Clear DM, existing nodes will be erased.
  add        <.dot file> Add a Schedule from input file to DM,
                         nodes with identical hashes (names) on
                         the DM will be ignored.
  overwrite  <.dot file> Overwrites all Schedules on DM with
                         the one in the input file, already
                         existing nodes on the DM will be erased.
  remove     <.dot file> Removes the schedule in the input file
                         from the DM, nodes with hashes (names)
                         not present on the DM will be ignored
  keep       <.dot file> Removes everything BUT the schedule in
                         the input file from the DM, nodes with
                         hashes (names) not present on the DM
                         will be ignored.
  rawvisited             Show 'visited' (1) / 'not visited' (0)
                         for all nodes.
  chkrem     <.dot file> Checks if all patterns in given dot can
                         be removed safely
  -n                     No verify, status will not be read after
                         operation
  -o         <.dot file> Specify output file name, default is
                         'download.dot'
  -s                     Show Meta Nodes. Download will not only
                         contain schedules, but also queues, etc.
  -v                     Verbose operation, print more details
  -d                     Debug operation, print everything
  -f                     Force, overrides the safety check for
                         clear, remove, overwrite and keep
\end{lstlisting}
