\documentclass[12pt,a4paper]{report}
% Language: English
\pdfminorversion=7
\usepackage[pdftex]{graphicx}
\usepackage{changepage}
\usepackage{xcolor}
\usepackage{listings}

\lstdefinestyle{dotfiles}{
  escapeinside={(*@}{@*)}, % (*@\label{mylabel}@*)
  numbers=left,
  stepnumber=1,
  numberstyle=\tiny,
  numbersep=10pt,
  captionpos=b,
  belowcaptionskip=1\baselineskip,
  breaklines=true,
  keepspaces=true,
  columns=flexible,
  language=C,
  showstringspaces=false,
  basicstyle=\scriptsize\ttfamily,
  keywordstyle=\color{green!40!black},
  commentstyle=\itshape\color{purple!40!black},
  identifierstyle=\color{blue},
  stringstyle=\color{red},
  tabsize=2,
  morekeywords={digraph, graph, subgraph, edge, node, color, style, shape, fillcolor},
}

\newcommand{\ry}{\rotatebox{90}}
\begin{document}

\begin{titlepage}
\vspace{2cm}
\begin{center}
\Huge{Tests for the Datamaster}

\Large{Martin Skorsky}

\Large{Last change: 2021-07-22}
\end{center}
\vfill
\end{titlepage}

\tableofcontents

\chapter{Overview - What is tested}
All tests are on branch \texttt{dm-fallout-tests}. The tests run with \texttt{make} or \texttt{make all} in folder \texttt{modules/ftm/tests}.
To compile \texttt{libcarpedm} use \texttt{make prepare}. This runs \texttt{make clean} and \texttt{make} in folder \texttt{modules/ftm/ftmx86}.

Each test has a seperate target in the \texttt{Makefile}.
\begin{table}
\caption{Which test tests what}
%\begin{center}
\centering
\begin{tabular}[t]{|l|c|c|c|c|c|c|c|c|}
\hline
Test               & \ry{Tools} & \ry{libcarpedm} & \ry{firmware} & \ry{uses Python} & \ry{common} & \ry{make} & \ry{checks result} \\ \hline
addDownloadCompare &   T        &   T             &   T           &   x              &   x         &   -       &   -                \\ \hline
bpcStart           &   -        &   T             &   x           &   x              &   x         &   -       &   x                \\ \hline
dmCmd              &   T        &   x             &   x           &   x              &   -         &   -       &   x                \\ \hline
dmPerformance      &   x        &   T             &   x           &   -              &   -         &   x       &   -                \\ \hline
dmThreads          &   x        &   x             &   T           &   -              &   -         &   x       &   -                \\ \hline
fid7               &   -        &   x             &   T           &   x              &   x         &   -       &   x                \\ \hline
full\_test         &   -        &   T             &   x           &   x              &   -         &   -       &   x                \\ \hline
pps                &   -        &   T             &   x           &   x              &   -         &   -       &   -                \\ \hline
safe2remove        &   x        &   T             &   T           &   -              &   -         &   x       &   x                \\ \hline
singleEdgeTest     &   -        &   T             &   -           &   -              &   -         &   -       &   x                \\ \hline
schedules          &   x        &   x             &   T           &   -              &   -         &   x       &   -                \\ \hline
\end{tabular}
%\end{center}
\end{table}
\chapter{The Tests}
\section{addDownloadCompare}
\texttt{addDownloadCompare} test that a schedule is equivalent to the schedule which is downloaded form the datamaster firmware.
The test steps are: clear the datamaster, add a schedule, start all pattern, download the schedule, compare both schedules with \texttt{scheduleCompare}.
There should be no difference with \texttt{scheduleCompare}. Each test case uses a different schedule.

This test requires \texttt{scheduleCompare} to be installed. This tool checks that two dot-files represent the same schedule. For installation use branch
origin/dm-analysis. The tool is build with make in folder modules/ftm/analysis/scheduleCompare/main/. It is installed with sudo make install in the same folder.
\section{bpcStart}
\texttt{bpcStart} tests the implementation of the beam process chain start flag in libcarpedm.
The test schedule sends two timing messages with \texttt{bpcstart=True} and \texttt{bpcstart=1}.
With \texttt{saft-ctl snoop} it is checked that the timing messages contain the correct setting.
In addition with \texttt{dm-sched} the dumped schedule is checked for the bpcstart flag.
\section{dmCmd}
\texttt{dmCmd} contains Python test scripts which contain Python unit tests for the tool dm-cmd. Each unit test calls dm-cmd
with commands and options and checks the result with the output on stdout and stderr. There are also negative tests with an invalid command
line. These tests are successful when the response is the correct error message and not a core dump.
\section{dmPerformance}
\texttt{dmPerformance} tests the performance improvements in libcarpedm.
The test starts a schedule on a clean data master, checks if some part of the schedule is removable, removes it and
then adds another schedule. This is done for a small schedule and a larger schedule. The test is ok if all commands
work. There is no check for this.
\section{dmThreads}
\texttt{dmThreads} tests the firmware with 4 threads. The same test for 6 threads fails due to a bug in the
firmware. For each thread a pattern with one block and one timing message is loaded.
\section{fid7}
\texttt{fid7} tests the fix for the format id 7 bug.
\section{full\_test}
\texttt{full\_test} is a collection of nine tests for scenarios with schedules, static and dynamic. The schedules are added and modified. The
status of queues is checked. Schedules are not started.
\subsection{How to run the tests}
Currently there are 9 automated tests for the data master. The tests are driven by a Python script \texttt{dm\_testman.py}.

The static tests just add and remove patterns. The dynamic test in addition start the patterns.
\subsection{Prerequisites}
This test framework requires Python 3.6m or higher. A data master has to be accessible from the host.
The source for the tests is at

Project: \texttt{https://github.com/GSI-CS-CO/bel\_projects/},

Path: \texttt{modules/ftm/ftmx86/full\_test/}.

For each test the data master is halted and cleared. All patterns are removed.

\subsection{Structure of Description}
\begin{enumerate}
	\item Purpose of Test

	What is the objective of this test?
	\item Prerequisites of Test

	What is the setting of the test?
	\item Test Actions

	List the actions of the test. This includes the graphs of the test pattern.
	\item Success Criteria

	What is checked to state a successful test?
\end{enumerate}
\subsection{Static Tests}
\subsubsection{Basic}
\begin{enumerate}
	\item Purpose of Test

	This test uses \texttt{dm-sched add} and \texttt{dm-sched remove}. The pattern Figure~\ref{fig:Pattern_for_the_static_basic_test}
	is loaded into data master and removed afterwards.
	\item Test Actions

	On a cleared data master the test pattern is added with \texttt{dm-sched add}. With \texttt{dm-sched status} it is checked
	that 24 nodes with the expected names are available. The test pattern is removed with \texttt{dm-sched remove}. At the
	end \texttt{dem-sched status} is used to check that no pattern is present on the data master.

    \begin{figure}
        \centering
        \includegraphics{TestPattern/static_basic.pdf}
        \caption{Pattern for the static basic test}
        \label{fig:Pattern_for_the_static_basic_test}
    \end{figure}
	\item Success Criteria

	The test is successful if no pattern is loaded. Checked with \texttt{dm-sched status}
\end{enumerate}
\subsubsection{Coupling}
\begin{enumerate}
	\item Purpose of Test

	This test enlarges an existing pattern with a second pattern with edges into the first
	pattern. See Figure~\ref{fig:Pattern_for_the_static_coupling_test} for the test patterns.
	\item Test Actions

	First, a pattern with three nodes is added. In a second step a pattern with additional three nodes is added.
	This pattern contains edges into the first pattern.
    \begin{figure}
        \centering
        \includegraphics{TestPattern/static_coupling1.pdf}
        \includegraphics{TestPattern/static_coupling2.pdf}
        \caption{Pattern for the static coupling test before and after coupling}
        \label{fig:Pattern_for_the_static_coupling_test}
    \end{figure}
	\item Success Criteria

	After adding the two patterns the status is checked with \texttt{dm-sched status}. The resulting \texttt{download.dot} is
	compared to an expected dot-file.
\end{enumerate}
\subsubsection{Priority and Type}
\begin{enumerate}
	\item Purpose of Test

The test checks the relative and the absolute time values for two nodes in a four node pattern.
See Figure~\ref{fig:Pattern_for_the_static_priority_and_type_test} for the test pattern
and \ref{fig:Pattern_for_the_static_priority_and_type_test_with_meta_nodes} for the test pattern with meta nodes,
displaying the priority queues.
	\item Test Actions

	Add the pattern, check the relative time values. Clear the data master. Add the pattern again and check the absolute time values.
    \begin{figure}
        \centering
        \includegraphics{TestPattern/static_prio_and_type.pdf}
        \caption{Pattern for the static priority and type test}
        \label{fig:Pattern_for_the_static_priority_and_type_test}
    \end{figure}
    \begin{figure}
        \centering
        \includegraphics*[height=0.95\textheight,keepaspectratio]{TestPattern/static_prio_and_type_meta.pdf}
        \caption{Pattern for the static priority and type test with meta nodes}
        \label{fig:Pattern_for_the_static_priority_and_type_test_with_meta_nodes}
    \end{figure}
	\item Success Criteria

	Two checks of the time values with \texttt{dm-cmd rawqueue}.
\end{enumerate}
\subsection{Dynamic Tests}
\subsubsection{Async}
\begin{enumerate}
	\item Purpose of Test

	For a schedule asynchronous clear a block, change the destination to a timing message and check that all nodes are visited.

	See Figure~\ref{fig:Pattern_for_the_dynamic_async_test} for the test pattern.
	\item Test Actions

	Upload test schedule and start pattern $LOOP$. Check with \texttt{dm-cmd rawvisited}. The
	nodes 'BLOCK\_B', 'BLOCK\_LOOP', 'CMD\_LOOP' are visited. Lock pattern 'B' (this locks 'BLOCK\_B'), check this with
	\texttt{showlocks}. Clear pattern 'B' which clears the queues of 'BLOCK\_B', check this with \texttt{rawqueue}.
	Change schedule with flow command. Destination of 'BLOCK\_B' is now 'MSG\_A' for one message. Check this
	with \texttt{rawqueue}. Unlock pattern 'B', wait for $1.2$ seconds and then check with \texttt{rawvisited} that all
	nodes are visited and no blocks are locked (with \texttt{showlocks}).
    \begin{figure}
        \centering
        \includegraphics{TestPattern/dynamic_async.pdf}
        \caption{Pattern for the dynamic async test}
        \label{fig:Pattern_for_the_dynamic_async_test}
    \end{figure}
	\item Success Criteria

	Check that node 'MSG\_A' is visited after changing the destination of the flow.
\end{enumerate}
\subsubsection{Basic - Run CPU 0 single}
\begin{enumerate}
	\item Purpose of Test

    Add the test schedule to the data master, start patterns and check which nodes were visited.

	See Figure~\ref{fig:Pattern_for_the_dynamic_run_CPU_0_single_test} for the test pattern.
	\item Test Actions

    First check that no block is visited. Start pattern 'IN0'. Check that 'BLOCK\_A' and 'BLOCK\_IN0' are visited.
    Start pattern 'IN1'. Check that in addition 'BLOCK\_B' and 'BLOCK\_IN1' are visited.
    \begin{figure}
        \centering
        \includegraphics{TestPattern/dynamic_basic_run_cpu0_single.pdf}
        \caption{Pattern for the dynamic run CPU 0 single test}
        \label{fig:Pattern_for_the_dynamic_run_CPU_0_single_test}
    \end{figure}
	\item Success Criteria

	In the end all blocks are visited.
\end{enumerate}
\subsubsection{Basic - Run all single}
\begin{enumerate}
	\item Purpose of Test

    Run a very basic schedule on all four CPUs.

	See Figure~\ref{fig:Pattern_for_the_dynamic_run_all_test} for the test pattern.
	\item Test Actions

	Add a schedule with four blocks to the data master. For each CPU, check that no block is visited. Then start
	a pattern for one CPU and check that the block specific for this pattern is visited.
    \begin{figure}
        \centering
        \includegraphics{TestPattern/dynamic_basic_run_all_single.pdf}
        \caption{Pattern for the dynamic run all test}
        \label{fig:Pattern_for_the_dynamic_run_all_test}
    \end{figure}
	\item Success Criteria

	For each pattern the correct CPU is used.
\end{enumerate}
\subsubsection{Basic - Start Stop Abort}
\begin{enumerate}
	\item Purpose of Test

    First part: Start and abort a pattern. Second part: Start and stop a pattern.

	See Figure~\ref{fig:Pattern_for_the_dynamic_start_stop_abort_test} for the test pattern.
	\item Test Actions

    Add a schedule, check that CPU 0 is idle and then start pattern 'IN\_C0'. Check that the pattern is running.
    Abort the pattern 'IN\_C0'. Check the visited nodes for the pattern. The second part is similar.
    Add the same schedule, check that CPU 0 is idle and then start pattern 'IN\_C0'. Check that the pattern is running.
    Stop the pattern 'IN\_C0'. Check the visited nodes for the pattern.
    \begin{figure}
        \centering
        \includegraphics{TestPattern/dynamic_basic_start_stop_abort.pdf}
        \caption{Pattern for the dynamic start stop abort test}
        \label{fig:Pattern_for_the_dynamic_start_stop_abort_test}
    \end{figure}
	\item Success Criteria

	Check that pattern is correctly aborted (rawstatus RUN is 0 immediately) or
	stopped (rawstatus RUN is 1 immediately, but 0 after 1.5 seconds).
\end{enumerate}
\subsubsection{Branch - Single}
\begin{enumerate}
	\item Purpose of Test

	Test that the flow command switches from one block to another.

	See Figure~\ref{fig:Pattern_for_the_dynamic_branch_single_test} for the test pattern.
	\item Test Actions

	Add a schedule, start the pattern 'IN\_C0'. After checking that nodes 'BLOCK\_IN0' and 'BLOCK\_A' are visited,
        change the flow with the flowpattern command at pattern 'IN\_C0' from 'A' to 'B'. Check that the flowpattern
        command is in the low priority queue and then start the pattern 'IN\_C0'. Check that the flowpattern
        command is processed in the low priority queue and node 'BLOCK\_B' is visited.
    \begin{figure}
        \centering
        \includegraphics{TestPattern/dynamic_branch_single.pdf}
        \caption{Pattern for the dynamic branch single test}
        \label{fig:Pattern_for_the_dynamic_branch_single_test}
    \end{figure}
	\item Success Criteria

	Changing flow from 'BLOCK\_A' to 'BLOCK\_B' works.
\end{enumerate}
\subsubsection{Coupling}
\begin{enumerate}
	\item Purpose of Test

	Test not working, needs set up.
	\item Test Actions
	\item Success Criteria
\end{enumerate}
\subsubsection{Loop}
\begin{enumerate}
	\item Purpose of Test

        Test loop with flow initializer.

	See Figure~\ref{fig:Pattern_for_the_dynamic_loop_test} for the test pattern.
	\item Test Actions

        Add a schedule and start pattern 'IN\_A'. Check the visited nodes. Nodes 'INIT\_A0', 'BLOCK\_LOOP', and 'BLOCK\_EXIT'
        should be visited. Start pattern 'IN\_B' and check the visited nodes. All nodes should be visited.
    \begin{figure}
        \centering
        \includegraphics{TestPattern/dynamic_loop.pdf}
        \caption{Pattern for the dynamic loop test}
        \label{fig:Pattern_for_the_dynamic_loop_test}
    \end{figure}
	\item Success Criteria

	In the end all blocks are visited.
\end{enumerate}
\subsubsection{Switch}
\begin{enumerate}
	\item Purpose of Test

	Test not working, needs set up.
	\item Test Actions
	\item Success Criteria
\end{enumerate}

\section{messageCounter} This test is obsolete.
\texttt{messageCounter} is a basic test for the Python-Wrapper of libcarpedm.
\section{pps}
\texttt{pps} (pulse per second) is a basic test with a schedule which sends two timing messages every second.
\section{safe2remove}
\texttt{safe2remove} tests to remove a pattern from a running schedule. Test steps:
\begin{enumerate}
\item Clear data master
\item Add schedule
\item Start all patterns
\item Check removal of one pattern, should fail while pattern is running.
\item Abort pattern
\item Check removal of this pattern, should be valid, since pattern is not running.
\item Remove this pattern.
\item Check status of remaining schedule.
\end{enumerate}
These test steps are applied to a bunch of schedules.
\section{singleEdgeTest}
\texttt{singleEdgeTest} tests the validation of combinations of node types and edge types. This uses libcarepedm only. There are 1936 schedules
with at least one edge and two nodes tested. Some schedule are enlarged with meta nodes.

For this, the combinations of a node and an outgoing edge and,
respectively, a node and an incoming edge.

The tables are compiled from source validation.cpp.

Classification of nodes:
\begin{enumerate}
\item Meta nodes: qinfo, qbuf, listdst
\item Event nodes: tmsg, switch, flow, flush, noop, waits
\item Command nodes: flow, flush, noop, wait
\end{enumerate}

\begin{table}
\begin{adjustwidth}{-25mm}{}
\caption{Schedule -- Valid edge types per node type}
\begin{tabular}[t]{|l|c|c|c|c|c|c|c|c|c|c|c|}

\hline
Edge Type & \multicolumn{11}{c|}{Node Type - Out-Edge, first node} \\
\hline
                 & block & blockalign & flow & flush & listdst & noop & qbuf & qinfo & switch & tmsg & wait \\
\hline
defdst           & 0..1  & 0..1       & 0..1 & 1     & --      & 0..1 & --   & --    & 0..1   & 1    & 0..1 \\
altdst           & 0..10 & 0..10      & --   & --    & --      & --   & --   & --    & --     & --   & --   \\
listdst          & 0..1  & 0..1       & --   & --    & --      & --   & --   & --    & --     & --   & --   \\
baddefdst        & --    & --         & --   & --    & --      & --   & --   & --    & --     & --   & --   \\
target (Switch)  & --    & --         & --   & --    & --      & --   & --   & --    & 0..1   & --   & --   \\
target (Command) & --    & --         & 0..1 & 0..1  & --      & 0..1 & --   & --    & --     & --   & 0..1 \\
flowdst          & --    & --         & 0..1 & --    & --      & --   & --   & --    & --     & --   & --   \\
flushovr         & --    & --         & --   & 0..1  & --      & --   & --   & --    & --     & --   & --   \\
switchdst        & --    & --         & --   & --    & --      & --   & --   & --    & 0..1   & --   & --   \\
meta             & --    & --         & --   & --    & --      & --   & --   & 2     & --     & --   & --   \\
priolo           & 0..1  & 0..1       & --   & --    & --      & --   & --   & --    & --     & --   & --   \\
priohi           & 0..1  & 0..1       & --   & --    & --      & --   & --   & --    & --     & --   & --   \\
prioil           & 0..1  & 0..1       & --   & --    & --      & --   & --   & --    & --     & --   & --   \\
dynid x          & --    & --         & --   & --    & --      & --   & --   & --    & --     & --   & --   \\
dynpar0          & --    & --         & --   & --    & --      & --   & --   & --    & --     & 0..1 & --   \\
dynpar1          & --    & --         & --   & --    & --      & --   & --   & --    & --     & 0..1 & --   \\
dyntef           & --    & --         & --   & --    & --      & --   & --   & --    & --     & --   & --   \\
dynres           & --    & --         & --   & --    & --      & --   & --   & --    & --     & --   & --   \\
any edge         & 0..15 & 0..15      & 1..3 & 1..3  & 0..     & 1..2 & 0..  & 2     & 1..3   & 1..3 & 1..2 \\
\hline
\end{tabular}
\end{adjustwidth}
\end{table}
The last row combines
\begin{enumerate}
\item the check for childless nodes (events and qinfo must have childs),
\item the sum of min and max cardinalities for the detailed edge types.
\end{enumerate}

\begin{table}
\begin{adjustwidth}{-25mm}{}
\caption{Schedule -- Valid edge types per node type}
\begin{tabular}[t]{|l|c|c|c|c|c|c|c|c|c|c|c|}

\hline
Edge Type & \multicolumn{11}{c|}{Node Type - In-Edge, second node} \\
\hline
                 & block & blockalign & flow  & flush & listdst & noop  & qbuf & qinfo & switch & tmsg & wait  \\
\hline
defdst           & 0..1  & 0..1       & 0..1  & 0..1  & --      & 0..1  & --   & --    & 0..1   & 0..1  & 0..1  \\
altdst           & 0..10 & 0..10      & 0..10 & 0..10 & --      & 0..10 & --   & --    & 0..10  & 0..10 & 0..10 \\
listdst          & 0..1  & 0..1       & --    & --    & --      & --    & --   & --    & --     & --    & --    \\
baddefdst        & --    & --         & --    & --    & --      & --    & --   & --    & --     & --    & --    \\
target (Switch)  & 0..1  & 0..1       & --    & --    & --      & --    & --   & --    & --     & --    & --    \\
target (Command) & 0..1  & 0..1       & --    & --    & --      & --    & --   & --    & --     & --    & --    \\
flowdst          & 0..1  & 0..1       & 0..1  & 0..1  & --      & 0..1  & --   & --    & 0..1   & 0..1  & 0..1  \\
flushovr         & --    & --         & --    & 0..1  & --      & --    & --   & --    & --     & --    & --    \\
switchdst        & 0..1  & 0..1       & 0..1  & 0..1  & --      & 0..1  & --   & --    & 0..1   & 0..1  & 0..1  \\
meta             & --    & --         & --    & --    & --      & --    & 1    & --    & --     & --    & --    \\
priolo           & --    & --         & --    & --    & --      & --    & --   & 0..1  & --     & --    & --    \\
priohi           & --    & --         & --    & --    & --      & --    & --   & 0..1  & --     & --    & --    \\
prioil           & --    & --         & --    & --    & --      & --    & --   & 0..1  & --     & --    & --    \\
dynid x          & --    & --         & --    & --    & --      & --    & --   & --    & --     & --    & --    \\
dynpar0          & 0..1  & 0..1       & 0..1  & 0..1  & --      & 0..1  & --   & --    & 0..1   & 0..1  & 0..1  \\
dynpar1          & 0..1  & 0..1       & 0..1  & 0..1  & --      & 0..1  & --   & --    & 0..1   & 0..1  & 0..1  \\
dyntef           & --    & --         & --    & --    & --      & --    & --   & --    & --     & --    & --    \\
dynres           & --    & --         & --    & --    & --      & --    & --   & --    & --     & --    & --    \\
any edge         & 0..18 & 0..18      & 0..15 & 0..16 & 0       & 0..15 & 1    & 0..1  & 0..15  & 0..15 & 0..15 \\
\hline
\end{tabular}
\end{adjustwidth}
\end{table}

Open questions:
\begin{enumerate}
\item Is the list of node types complete?
\item Is the list of edge types complete?
\item Node type listdst: no rules for in- or out-edges defined. Is this correct?
\item Edge types dynid x, dyntef, dynres: are these edge types used?
\item How to distinguish edge types target (Switch) and target (Command)?
\item Node type flush and flow: these nodes may have priority queues (lo, hi, il). Why are the queues allowed?
There are no meta nodes generated for these queues. Why?
\item The rules allow two edges from a block to a bufferlist with edge types priolo and priohi.
Should the rules be fixed for this invalid schedule?
\item Is an edge of type defdst necessary for a node of type flow? For node type flush it is necessary.
\item There are no ConstellationRules for nodes of type listdst and qbuf. Is this correct? Yes! Node types must be childless.
\item Nodes types listdst and qbuf must be childless.
\end{enumerate}


\section{schedules}
\texttt{schedules} collects schedules which are started. Tests that the schedules are compiled and loaded.
\chapter{Common Components}
\section{dm\_testbench.py}
\texttt{dm\_testbench.py} is a collection of Python functions for use in other test scripts.
\begin{enumerate}
\item startpattern(data\_master, pattern\_file)
    Connect to the given data master and load the pattern file (dot format).
    The data master is halted, cleared, and statistics is reset.
    Search for the first pattern in the data master with 'dm-sched' and start it.
\end{enumerate}
\end{document}
