\documentclass[12pt,a4paper]{report}
% Language: English
\pdfminorversion=7
\usepackage[pdftex]{graphicx}
\usepackage{changepage}
\usepackage{xcolor}
\usepackage{listings}

\lstdefinestyle{dotfiles}{
  escapeinside={(*@}{@*)}, % (*@\label{mylabel}@*)
  numbers=left,
  stepnumber=1,
  numberstyle=\tiny,
  numbersep=10pt,
  captionpos=b,
  belowcaptionskip=1\baselineskip,
  breaklines=true,
  keepspaces=true,
  columns=flexible,
  language=C,
  showstringspaces=false,
  basicstyle=\scriptsize\ttfamily,
  keywordstyle=\color{green!40!black},
  commentstyle=\itshape\color{purple!40!black},
  identifierstyle=\color{blue},
  stringstyle=\color{red},
  tabsize=2,
  morekeywords={digraph, graph, subgraph, edge, node, color, style, shape, fillcolor},
}

\newcommand{\ry}{\rotatebox{90}}
\begin{document}

\begin{titlepage}
\vspace{2cm}
\begin{center}
\Huge{Documentation Data Master Test System}

\Large{Martin Skorsky}

\Large{Last change: 2021-05-12}
\end{center}
\vfill
\end{titlepage}

\tableofcontents

\chapter{Hardware}
\section{SuperMicro fel0069}
The data master for the test system is hosted on SuperMicro fel0069 with two PEXARIA5d \texttt{(fel0069.acc.gsi.de)}.
One of these is the data master. The other one is used to analyse the timing messages with snoop.
The SuperMicro is accessible with \texttt{ssh root@fel0069.acc} via ASL cluster or other hosts.
The management interface (ILO) is accessible via \texttt{https://fel0069i64.acc}.
\begin{enumerate}
\item fel0069.acc.gsi.de
\begin{itemize}
\item IP: 140.181.169.146
\item Location: BG2.009, Rack BG2A.A9, Slot 28
\end{itemize}
\item pexaria248t (dev/wbm0), Data master
\begin{itemize}
\item IP: 192.168.131.184
\item MAC: 00:26:7b:00:08:0b
\item Name: pexaria248t
\item CID: 55 0113 0012 0
\item PEXARIA5d, Serie EE
\end{itemize}
\item pexaria305t (dev/wbm1), Timing receiver for snoop
\begin{itemize}
\item IP: 192.168.131.241
\item MAC: 00:26:7b:00:08:44
\item Name: pexaria305t
\item CID: 55 0113 0069 4
\item PEXARIA5d, Serie EE
\end{itemize}
\end{enumerate}
After power on, set the IP addresses of the two pexarias. This host has no BootP service.

The SuperMicro is configured for PXE boot and nfsinit with links (follwing
https://www-acc.gsi.de/wiki/Timing/Intern/TimingSystemHowToHintsForFECS)
\begin{enumerate}
\item On ASL cluster links in folder
\item On ASL cluster links in folder /common/export/nfsinit/fel0069/ for nfsinit.

\end{enumerate}

\section{White Rabbit Switch nwt0473m66}
Location: BG2.009, Rack BG2A.A9, Slot 29
blank configuration. Not an access switch or distribution switch
Access via tsl001, like other switches.
\chapter{Software}
\section{Access the Data Master}
\texttt{dm-cmd tcp/fel0069.acc}

\texttt{dm-sched tcp/fel0069.acc}

\section{Remote Snoop of Timing Messages}
snoop per remote ssh / set up of ssh without password: use public key.
\begin{verbatim}
scp -p your_pub_key.pub user@host:
ssh user@host
host$ cat id_dsa.pub >> ~/.ssh/authorized_keys
\end{verbatim}
snoop with Python3
\end{document}
