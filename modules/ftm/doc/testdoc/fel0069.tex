\documentclass[12pt,a4paper]{report}
% Language: English
\pdfminorversion=7
\usepackage[pdftex]{graphicx}
\usepackage{changepage}
\usepackage{xcolor}
\usepackage{listings}
\usepackage{xurl}

\lstdefinestyle{dotfiles}{
  escapeinside={(*@}{@*)}, % (*@\label{mylabel}@*)
  numbers=left,
  stepnumber=1,
  numberstyle=\tiny,
  numbersep=10pt,
  captionpos=b,
  belowcaptionskip=1\baselineskip,
  breaklines=true,
  keepspaces=true,
  columns=flexible,
  language=C,
  showstringspaces=false,
  basicstyle=\scriptsize\ttfamily,
  keywordstyle=\color{green!40!black},
  commentstyle=\itshape\color{purple!40!black},
  identifierstyle=\color{blue},
  stringstyle=\color{red},
  tabsize=2,
  morekeywords={digraph, graph, subgraph, edge, node, color, style, shape, fillcolor},
}

\newcommand{\ry}{\rotatebox{90}}
\begin{document}

\begin{titlepage}
\vspace{2cm}
\begin{center}
\Huge{Documentation Data Master Test System}

\Large{Martin Skorsky}

\Large{Last change: 2022-01-27}
\end{center}
\vfill
\end{titlepage}

\tableofcontents

\chapter{Hardware}
\section{SuperMicro fel0069}
The data master for the test system is hosted on SuperMicro fel0069 with two PEXARIA5d \texttt{(fel0069.acc.gsi.de)}.
One of these is the data master. The other one is used to analyse the timing messages with snoop.
The SuperMicro is accessible with \texttt{ssh root@fel0069.acc} via ASL cluster or other hosts.
The management interface (ILO) is accessible via \url{https://fel0069i64.acc}.
\begin{enumerate}
\item fel0069.acc.gsi.de
\begin{itemize}
\item IP: 140.181.169.146 (in Hex für PXE Boot: 8C.B5.A9.92)
\item MAC ILO: ac:1f:6b:71:fe:8f, IPv4 ILO: 140.181.169.147
\item MAC Lan1: ac:1f:6b:72:02:2c
\item MAC Lan2: ac:1f:6b:72:02:2d
\item Location: BG2.009, Rack BG2A.A9, Slot 28
\end{itemize}
\item pexaria248t (dev/wbm0), Data master
\begin{itemize}
\item IP: 192.168.131.184
\item MAC: 00:26:7b:00:08:0b
\item Name: pexaria248t
\item CID: 55 0113 0012 0
\item PEXARIA5d, Serie EE
\end{itemize}
\item pexaria305t (dev/wbm1), Timing receiver for snoop
\begin{itemize}
\item IP: 192.168.131.241
\item MAC: 00:26:7b:00:08:44
\item Name: pexaria305t
\item CID: 55 0113 0069 4
\item PEXARIA5d, Serie EE
\end{itemize}
\end{enumerate}
After power on, set the IP addresses of the two pexarias. This host has no BootP service.

The SuperMicro is configured for PXE boot and nfsinit with links (following
\url{https://www-acc.gsi.de/wiki/Timing/Intern/TimingSystemHowToHintsForFECS}).
\begin{enumerate}
\item On ASL cluster links in folder /common/tftp/csco/pxe/pxelinux.cfg for PXE boot:
8CB5A992 $\to$ fel0069,
fel0069 $\to$ scuxl.fallout
\item On ASL cluster links in folder /common/export/nfsinit/fel0069/ for nfsinit.
\item On ASL cluster create folder /common/fesadata/data/fel0069/.
\end{enumerate}
Connections: network cable to acc network. For management: network cable to acc network.

After \texttt{reboot} of fel0069:
\begin{enumerate}
\item on acopc042 (or some other host/user which needs access to fel0069) copy the public ssh key to fel0069.
This has to be done for every user / host which needs access to fel0069 with ssh.\raggedright
\linebreak ssh-copy-id -i .ssh/id\_rsa.pub root@fel0069.acc

\item test (check before automated testing starts)\raggedright
\linebreak ssh -t root@fel0069.acc.gsi.de "saft-ctl tr1 -xv snoop 0 0 0 3"\raggedright
\linebreak (snoop on tr1 for all events for 3 seconds)
\end{enumerate}

\section{White Rabbit Switch nwt0473m66}
Location: BG2.009, Rack BG2A.A9, Slot 29 \raggedright
\linebreak Configuration: blank. Not an access switch or distribution switch!
\linebreak Access via tsl001, like other switches.
\linebreak Name: nwt0473m66.timing.acc.gsi.de,
\linebreak IP: 192.168.21.219.
\linebreak Connections: fibre optic cable from wri2 to pexaria248t, fibre optic cable from wri3 to pexaria305t. Network cable to acc network for management.
\chapter{Software}
\section{Firmware Images}
Datamaster: build with \texttt{make ftm} in bel\_projects root folder. Current version:
\begin{verbatim}
Project     : ftm
Platform    : pexaria5 +db[12] +wrex1
FPGA model  : Arria V (5agxma3d4f27i3)
Source info : fallout-3295
Build type  : developer preview
Build date  : Wed Oct 28 16:45:33 CET 2020
Prepared by : Martin Skorsky <m.skorsky@gsi.de>
Prepared on : ACOPC042
OS version  : Linux Mint 19.3 Tricia, kernel 4.15.0-122-generic
Quartus     : Version 18.1.0 Build 625 09/12/2018 SJ Standard
                                                        Edition

  176fccd2 pmc: changed seed
  d3ab3c45 vetar2a-ee-butis: changed seed
  5419e84e microtca: changed seed
  c0d44bd6 Merge pull request #260 from
                                  GSI-CS-CO/dm-fallout-merge-v2
  50444f38 scu4: changed seed

Detecting Firmwares ...

Found 4 RAMs, 4 holding a Firmware ID


********************
* RAM @ 0x04120000 *
********************
UserLM32
Stack Status:
Project     : ftm
Version     : 7.0.1
Platform    :
Build Date  : Wed Oct 28 16:45:19 CET 2020
Prepared by : martin Martin Skorsky <m.skorsky@gsi.de>
Prepared on : ACOPC042
OS Version  : Linux Mint 19.3 Tricia  Linux 4.15.0-122-generic
                                                         x86_64
GCC Version : lm32-elf-gcc(GCC)4.5.3 (build 190527-673a32-f3d6)
IntAdrOffs  : 0x10000000
SharedOffs  : 0x500
SharedSize  : 98304
FW-ID ROM will contain:

   176fccd2 pmc: changed seed
   d3ab3c45 vetar2a-ee-butis: changed seed
   5419e84e microtca: changed seed
   c0d44bd6 Merge pull request #260 from
                                GSI-CS-CO/dm-fallout-merge-v2
   50444f38 scu4: changed seed
*****
\end{verbatim}

Pexaria for snoop: \url{https://github.com/GSI-CS-CO/bel\_projects/releases/download/fallout-v6.0.1/pexarria5.rpd}

\section{Access the Data Master}
\texttt{dm-cmd tcp/fel0069.acc}
\linebreak
\texttt{dm-sched tcp/fel0069.acc}

\section{Remote Snoop of Timing Messages}
Snoop per remote ssh: Set up of ssh without password: use public key of user@host and transfer it to root@fel0069.acc:~/.ssh/authorized\_keys with \texttt{ssh-copy-id}.
Snoop with Python3: the tests using python3 / pytest read the command for snooping from environment variable SNOOP.
\begin{itemize}
\item Example for local environment: saft-ctl tr0 -xv snoop 0 0 0
\item Example for remote environment on fel0069.acc: ssh -t root@fel0069.acc 'saft-ctl tr1 -xv snoop 0 0 0' \linebreak
saftd on fel0069.acc monitors dev/wbm1 as tr1.
\end{itemize}
The tests add an additional parameter for the number of seconds to snoop.

\section{Required versions}
\begin{enumerate}
\item Python 3.6.9
\item pytest 6.1.2
\end{enumerate}

\section{Jenkins on tsl025.acc.gsi.de}
Update tsl025 with sudo apt update / upgrade.
Preparation of jenkins user on tsl025 for remote snoop.
\linebreak Switch to user jenkins on tsl025:
\linebreak sudo su -s /bin/bash jenkins
\linebreak ssh-keygen -t rsa -b 4096
\linebreak ssh-copy-id -i /var/lib/jenkins/.ssh/id\_rsa.pub root@fel0069
\linebreak ssh -t root@fel0069 'saft-ctl tr1 -xv snoop 0 0 0 3' (this is the test that the previous commands were successful)

\subsection{Jenkins Jobs on tsl025}
This Jenkins server is \url{http://tsl025:8080/}.
\begin{enumerate}
\item Build and install the tools (dm-cmd, dm-sched, libcarpedm)
\item Build and install scheduleCompare
\item check remote snoop
\item Run tests (13 out of 49 tests are skipped because of longer duration). Takes about 40 seconds. Useful during test developement.
\item Run all 49 tests. Takes up to 6 min. Normal test execution.
\end{enumerate}

\section{Install Image to Test on Datamaster}
This section describes the installation of \texttt{fallout-v6.1.2-rc1} on data master in test environment.
For the build, \texttt{acopc042} is used. Adopt both for a different release and build environment.

\subsection{Preparation: Build FTM Image}
\begin{verbatim}
# Preparation: quit saftd
saft-ctl tr0 quit
# switch to build folder:
cd ~/bel_projects/release
# update git repo
git pull
# switch to tagged version
git checkout fallout-v6.1.2-rc1
# build
## 1. clean up
git clean -x -f
cd ~/bel_projects/release/ip_cores/etherbone/
git clean -x -f
cd ~/bel_projects/release/ip_cores/saftlib/
git clean -x -f
cd -
git status
## 2. make and install
make
### sign kernel modules
~/scripts/sign-wishbone-modules.sh
sudo make install
### make and install etherbone
make etherbone
sudo make etherbone-install
### make and install saftlib
make saftlib
sudo make saftlib-install
sudo ldconfig
### make FTM image including FTM firmware
make ftm
# this produces:
# Project     : ftm
# Platform    : pexaria5 +db[12] +wrex1
# FPGA model  : Arria V (5agxma3d4f27i3)
## 3. Build DM Tools
cd ~/bel_projects/release/modules/ftm/ftmx86/
make
sudo make install
\end{verbatim}

\subsection{Flash on dal029 (data master develoment environment)}

Remark: datamaster for development environment is now on acopc042, dev/wbm1 ! Commands have to be changed accordingly.

On \texttt{dal029.acc.gsi.de} do (use \texttt{ssh root@dal029.acc.gsi.de}):
\begin{verbatim}
# 1. Preparation:
scp martin@acopc042:bel_projects/release/syn/gsi_pexarria5/ftm/ftm.rpd
                   ftm_v6.1.2-rc1.rpd
## stop saftd - error message if saftd is not running
saft-ctl tr0 quit
## stop datamaster
dm-cmd dev/wbm0 halt
## halt all lm32 CPUs
eb-reset dev/wbm0 cpuhalt 0xff
# 2. Flash - don't use eb-flash-secure.
# Included preparation not ok.
# Parameters -s and -w not supported
eb-flash -s 0x10000 -w 3 dev/wbm0 ftm_v6.1.2-rc1.rpd
# 3. Reboot
eb-reset dev/wbm0 fpgareset; reboot
\end{verbatim}
Now, if result of flash looks OK, update PXE boot (on ASL cluster).
\begin{verbatim}
ssh mskorsky@asl742.acc
cd /common/export/nfsinit/dal029/
rm 20_timing-rte
ln -s ../global/timing-rte-tg-fallout-v6.1.2-rc1 20_timing-rte
str-D # end ssh session
\end{verbatim}
Second reboot of dal069.
\begin{verbatim}
ssh root@dal069.acc
reboot
\end{verbatim}
Configure dal069 after reboot.
\begin{verbatim}
ssh root@dal029.acc
eb-console dev/wbm0
ip set 192.168.128.204
strg-D # end ssh session
\end{verbatim}
Finally: check that datamaster is up and run tests (do this on acopc042).
\begin{verbatim}
dm-cmd tcp/dal029.acc
# Run tests
# on acopc042: export DM=tcp/dal029.acc
cd ~/be_projects/test/module/ftm/tests
# Run quick tests (40 seconds)
make
# Run all tests (6 minutes)
OPTIONS=--runslow make
\end{verbatim}

\subsection{Flash on fel0069 (data master test environment)}
On \texttt{fel0069.acc.gsi.de} do (use \texttt{ssh root@fel0069.acc.gsi.de}):
\begin{verbatim}
# 1. Preparation:
Version 6.1.2 / FTM 7.0.1:
scp martin@acopc042.gsi.de:bel_projects/release/syn/gsi_pexarria5/ftm/ftm.rpd
           ftm_v6.1.2-rc1.rpd
wget https://github.com/GSI-CS-CO/bel_projects/releases/download
            /fallout-v6.1.2-rc1/falloutv6_1_2-rc1-pexarria5.rpd
Version 6.2.0 / FTM 8.0.0:
scp martin@acopc042.gsi.de:Documents/fel0069/ftm_722413ec_2021_11_02.rpd
           ftm_722413ec_2021_11_02.rpd
pexarria5: same as version 6.1.2.
## stop fesa: how to?
## stop saftd - error message if saftd is not running
saft-ctl tr1 quit
## stop datamaster
dm-cmd dev/wbm0 halt
## halt all lm32 CPUs on both pexarrias
eb-reset dev/wbm0 cpuhalt 0xff
eb-reset dev/wbm1 cpuhalt 0xff
# 2. Flash - don't use eb-flash-secure.
# Included preparation not ok.
# Parameters -s and -w not supported
eb-flash -s 0x10000 -w 3 dev/wbm0 'ftm rpd file'
eb-flash -s 0x10000 -w 3 dev/wbm1 'pexarria5 rpd file'
# 3. Reboot
eb-reset dev/wbm0 fpgareset; eb-reset dev/wbm1 fpgareset; reboot
\end{verbatim}
Now, if result of flash looks OK, update PXE boot (on ASL cluster).
\begin{verbatim}
ssh mskorsky@asl742.acc
cd /common/export/nfsinit/fel0069/
# adopt gateware timing rte
rm 20_timing-rte
ln -s ../global/timing-rte-tg-fallout-v6.1.2-rc1 20_timing-rte
# adopt DM tools, includes dm-cmd, dm-sched, libcarpedm
rm 40_generator_dev
ln -s ../global/dm-.... 40_generator_dev
str-D # end ssh session
\end{verbatim}
Second reboot of fel0029.
\begin{verbatim}
ssh root@fel0069.acc
reboot
\end{verbatim}
Configure fel0029 after reboot.
\begin{verbatim}
ssh-copy-id -i .ssh/id_rsa.pub root@fel0069.acc
ssh root@fel0069.acc
eb-console dev/wbm0
ip set 192.168.131.184
eb-console dev/wbm1
ip set 192.168.131.241
# Start saftd
saftd tr1:dev/wbm1
strg-D # end ssh session
\end{verbatim}
Finally: check that datamaster is up and run tests (do this on acopc042).
\begin{verbatim}
dm-cmd tcp/fel0069.acc
\end{verbatim}
Run tests on tsl025.acc: start Jenkins Jobs \url{http://tsl025:8080/}

\section{Open tasks}
\begin{enumerate}
\item Configure tsl025 as Jenkins-Slave of builder.acc.gsi.de
\item Add scripts / jobs to install a defined firmware / datamaster firmware on fel0069/dev/wbm0
\end{enumerate}

\end{document}
