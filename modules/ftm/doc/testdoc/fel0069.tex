\documentclass[12pt,a4paper]{report}
% Language: English
\pdfminorversion=7
\usepackage[pdftex]{graphicx}
\usepackage{changepage}
\usepackage{xcolor}
\usepackage{listings}
\usepackage{xurl}

\lstdefinestyle{dotfiles}{
  escapeinside={(*@}{@*)}, % (*@\label{mylabel}@*)
  numbers=left,
  stepnumber=1,
  numberstyle=\tiny,
  numbersep=10pt,
  captionpos=b,
  belowcaptionskip=1\baselineskip,
  breaklines=true,
  keepspaces=true,
  columns=flexible,
  language=C,
  showstringspaces=false,
  basicstyle=\scriptsize\ttfamily,
  keywordstyle=\color{green!40!black},
  commentstyle=\itshape\color{purple!40!black},
  identifierstyle=\color{blue},
  stringstyle=\color{red},
  tabsize=2,
  morekeywords={digraph, graph, subgraph, edge, node, color, style, shape, fillcolor},
}

\newcommand{\ry}{\rotatebox{90}}
\begin{document}

\begin{titlepage}
\vspace{2cm}
\begin{center}
\Huge{Documentation Data Master Test System}

\Large{Martin Skorsky}

\Large{Last change: 2021-05-12}
\end{center}
\vfill
\end{titlepage}

\tableofcontents

\chapter{Hardware}
\section{SuperMicro fel0069}
The data master for the test system is hosted on SuperMicro fel0069 with two PEXARIA5d \texttt{(fel0069.acc.gsi.de)}.
One of these is the data master. The other one is used to analyse the timing messages with snoop.
The SuperMicro is accessible with \texttt{ssh root@fel0069.acc} via ASL cluster or other hosts.
The management interface (ILO) is accessible via \url{https://fel0069i64.acc}.
\begin{enumerate}
\item fel0069.acc.gsi.de
\begin{itemize}
\item IP: 140.181.169.146
\item Location: BG2.009, Rack BG2A.A9, Slot 28
\end{itemize}
\item pexaria248t (dev/wbm0), Data master
\begin{itemize}
\item IP: 192.168.131.184
\item MAC: 00:26:7b:00:08:0b
\item Name: pexaria248t
\item CID: 55 0113 0012 0
\item PEXARIA5d, Serie EE
\end{itemize}
\item pexaria305t (dev/wbm1), Timing receiver for snoop
\begin{itemize}
\item IP: 192.168.131.241
\item MAC: 00:26:7b:00:08:44
\item Name: pexaria305t
\item CID: 55 0113 0069 4
\item PEXARIA5d, Serie EE
\end{itemize}
\end{enumerate}
After power on, set the IP addresses of the two pexarias. This host has no BootP service.

The SuperMicro is configured for PXE boot and nfsinit with links (follwing
\url{https://www-acc.gsi.de/wiki/Timing/Intern/TimingSystemHowToHintsForFECS}).
\begin{enumerate}
\item On ASL cluster links in folder /common/tftp/csco/pxe/pxelinux.cfg for PXE boot:
8CB5A992 $\to$ fel0069,
fel0069 $\to$ scuxl.fallout
\item On ASL cluster links in folder /common/export/nfsinit/fel0069/ for nfsinit.
\item On ASL cluster create folder /common/fesadata/data/fel0069/.
\end{enumerate}
Connections: network cable to acc network. For management: network cable to acc network.
\section{White Rabbit Switch nwt0473m66}
Location: BG2.009, Rack BG2A.A9, Slot 29 \raggedright
\linebreak Configuration: blank. Not an access switch or distribution switch!
\linebreak Access via tsl001, like other switches.
\linebreak Name: nwt0473m66.timing.acc.gsi.de,
\linebreak IP: 192.168.21.219.
\linebreak Connections: fibre optic cable from wri2 to pexaria248t, fibre optic cable from wri3 to pexaria305t. Network cable to acc network for management.
\chapter{Software}
\section{Firmware Images}
Datamaster: build with \texttt{make ftm} in bel\_projects root folder. Current version:
\begin{verbatim}
Project     : ftm
Platform    : pexaria5 +db[12] +wrex1
FPGA model  : Arria V (5agxma3d4f27i3)
Source info : fallout-3295
Build type  : developer preview
Build date  : Wed Oct 28 16:45:33 CET 2020
Prepared by : Martin Skorsky <m.skorsky@gsi.de>
Prepared on : ACOPC042
OS version  : Linux Mint 19.3 Tricia, kernel 4.15.0-122-generic
Quartus     : Version 18.1.0 Build 625 09/12/2018 SJ Standard
                                                        Edition

  176fccd2 pmc: changed seed
  d3ab3c45 vetar2a-ee-butis: changed seed
  5419e84e microtca: changed seed
  c0d44bd6 Merge pull request #260 from
                                  GSI-CS-CO/dm-fallout-merge-v2
  50444f38 scu4: changed seed

Detecting Firmwares ...

Found 4 RAMs, 4 holding a Firmware ID


********************
* RAM @ 0x04120000 *
********************
UserLM32
Stack Status:
Project     : ftm
Version     : 7.0.1
Platform    :
Build Date  : Wed Oct 28 16:45:19 CET 2020
Prepared by : martin Martin Skorsky <m.skorsky@gsi.de>
Prepared on : ACOPC042
OS Version  : Linux Mint 19.3 Tricia  Linux 4.15.0-122-generic
                                                         x86_64
GCC Version : lm32-elf-gcc(GCC)4.5.3 (build 190527-673a32-f3d6)
IntAdrOffs  : 0x10000000
SharedOffs  : 0x500
SharedSize  : 98304
FW-ID ROM will contain:

   176fccd2 pmc: changed seed
   d3ab3c45 vetar2a-ee-butis: changed seed
   5419e84e microtca: changed seed
   c0d44bd6 Merge pull request #260 from
                                GSI-CS-CO/dm-fallout-merge-v2
   50444f38 scu4: changed seed
*****
\end{verbatim}

Pexaria for snoop: \url{https://github.com/GSI-CS-CO/bel\_projects/releases/download/fallout-v6.0.1/pexarria5.rpd}

\section{Access the Data Master}
\texttt{dm-cmd tcp/fel0069.acc}
\linebreak
\texttt{dm-sched tcp/fel0069.acc}

\section{Remote Snoop of Timing Messages}
Snoop per remote ssh: \texttt{}. Set up of ssh without password: use public key of user@host and transfer it to root@fel0069.acc:~/.ssh/authorized\_keys.
\begin{verbatim}
scp your_pub_key.pub root@fel0069.acc:
ssh root@fel0069.acc
root@fel0069.acc$ cat id_dsa.pub >> ~/.ssh/authorized_keys
\end{verbatim}
snoop with Python3
\end{document}
