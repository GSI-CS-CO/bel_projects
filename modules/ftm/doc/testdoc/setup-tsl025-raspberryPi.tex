\documentclass[12pt,a4paper]{report}
% Language: English
\pdfminorversion=7
\usepackage[pdftex]{graphicx}
\usepackage{changepage}
\usepackage{xcolor}
\usepackage{listings}

\begin{document}

\begin{titlepage}
\vspace{2cm}
\begin{center}
\Huge{Set up tsl025 Jenkins Server on Raspberry Pi}

\Large{Martin Skorsky}

\Large{Last change: 2022-11-16 \\ Created: 2022-11-15}
\end{center}
\vfill
\end{titlepage}

\tableofcontents

\chapter{Install the Raspberian Image}
Install Raspberry Pi with Image writer:
https://howchoo.com/pi/install-raspberry-pi-os

Image: Bullseye 64Bit Lite (no Desktop) for Raspberry Pi 4 B.

Use settings in Raspberry imager to enable ssh.

Versions: python 3.9.2, Boost 1.74, gcc 10.2.1

\chapter{Maschine Setup}
Use raspi-config to set locale and timezone
\begin{verbatim}
sudo dpkg-reconfigure locales
\end{verbatim}

Configure time synchronization:
\begin{verbatim}
sudo nano /etc/systemd/timesyncd.conf
Entry: NTP=ntp1.acc.gsi.de
systemctl restart systemd-timesyncd.service
\end{verbatim}
Use sudo apt update / upgrade to update system.

\chapter{Install the bel\_projects Packages}
First install all packages needed for bel\_projects. Then build and install
Etherbone and EB Tools after installing Jenkis and running the first job.
\begin{verbatim}
sudo apt install automake autoconf
sudo apt install git
sudo apt install libsigc++-2.0-dev libglib2.0-dev
sudo apt install python3-pip
sudo apt install xsltproc
sudo apt install docbook-utils build-essential
sudo apt install libtool libreadline-dev
\end{verbatim}
Follow bel\_projects README for up to date instructions.

\chapter{Install Jenkins}
Use https://pimylifeup.com/jenkins-raspberry-pi/s to install jenkins.
\begin{verbatim}
sudo apt install openjdk-11-jre
sudo apt install openjdk-11-jdk-headless
sudo apt install jenkins
sudo apt install ant
sudo apt install junit junit4
sudo apt install python3-pytest
\end{verbatim}
After installation, Jenkins is started automatically by the installation.
Since the timeout for the start is too short, jenkins.service restarts all over. Use
\begin{verbatim}
sudo systemctl edit jenkins
\end{verbatim}
to configure the timeout. The command creates \\
\texttt{/etc/systemd/system/jenkins.service.d/override.conf}.
Set timeout for startup to 300 seconds.
\begin{verbatim}
[Service]
TimeoutStartSec=300
\end{verbatim}
For details see https://community.jenkins.io/t/job-for-jenkins-service-failed-because-a-timeout-was-exceeded/2443

Jenkins initial setup is required. An admin user has been created and a password generated.
Follow the instructions during setup and on \\
https://tsl025.acc.gsi.de:8080 for this initial setup.

The Jenkins installation creates a user jenkins. Add user jenkins to sudoers (i.e. add
jenkins to group sudo and configure that no password is needed).
\begin{verbatim}
su -s /bin/bash jenkins
sudo usermod jenkins -G sudo
sudo cd /etc/sudoers.d/
sudo cp 010_pi-nopasswd 010_jenkins-nopasswd
sudo chmod +w 010_jenkins-nopasswd
sudo nano 010_jenkins-nopasswd
sudo chmod -w 010_jenkins-nopasswd
\end{verbatim}

For convenience customize the profile of user jenkins with .bashrc,\\
.bash\_logout, .bash\_aliases.

Add an access token for your Jenkins user in the user profile. This is
needed for calls to the Jenkins API. Example:
\begin{verbatim}
curl -X POST -L --user $1 http://tsl025.acc.gsi.de:8080/ \
    job/$2/lastBuild/api/json | python3 -m json.tool > \
    buildResult_$2.json
\end{verbatim}
Here, \$1 is the access token and \$2 is the job name.

\chapter{Install Boost Libraries}
For the build of the data master tools we need three Boost libraries including
headers. These are in the -dev packages.
\begin{verbatim}
sudo apt install --install-suggests \
      libboost-serialization-dev \
      libboost-graph-dev     libboost-regex-dev
sudo apt install --install-suggests \
      libboost-serialization1.74-dev \
      libboost-graph1.74-dev libboost-regex1.74.0-dev
sudo apt install --install-suggests \
      libboost-serialization1.74.0 \
      libboost-graph1.74.0   libboost-regex1.74.0
\end{verbatim}

\chapter{Run Jenkins Jobs}
Copy the Jenkins job definitions (config.xml) from some backup to the folders
with the job names below folder \texttt{jobs}.
\begin{verbatim}
mkdir Build_and_load_FTM_firmware
mkdir build_schedule_compare
mkdir build_tools
mkdir 'check remote snoop'
mkdir Datamaster_Tests_dm-summer-update-2022_singleTest
mkdir Datamaster_Tests_dm-summer-update-2022
mkdir 'Java Test'
mkdir test_module_ftm_datamaster
mkdir test_module_ftm_datamaster_long_run
mkdir 'test_module_ftm_datamaster_long_run dm-fallout-tests-2'
\end{verbatim}
Restart Jenkins such that
Jenkins reads these definitions.

Start one of the Jenkins jobs which runs data master tests. This will fail on the first run,
but it clones the git repository. Inside such a repository of bel\_projects run the following
commands to build an install the data master tools and their prerequisites.
\begin{verbatim}
# in workspace folder of the job:
make
# for libetherbone and header file:
make etherbone
sudo make etherbone-install
# for eb-reset:
make tools
sudo make tools-install
\end{verbatim}
For remote snoop commands from the tests copy the ecdsa key for user jenkins to fel0069:
\begin{verbatim}
sudo su -s /bin/bash jenkins
ssh-keygen -t ecdsa
ssh-copy-id -i /var/lib/jenkins/.ssh/id\_ecdsa.pub \
      root@fel0069.acc.gsi.de
\end{verbatim}
On fell0069.acc.gsi.de copy the key from /etc/dropbear/authorized\_keys to .ssh/authorized\_keys
\begin{verbatim}
tail -1 /etc/dropbear/authorized_keys >> .ssh/authorized_keys
\end{verbatim}
The following command on tsl0025 checks that the ssh-copy-id was successful.
\begin{verbatim}
ssh root@fel0069 'saft-ctl tr1 -xv snoop 0 0 0 1'
\end{verbatim}

For the Java Test link junit4.jar and hamcrest-all.jar from /usr/share/ant/lib.
\end{document}
