\documentclass[12pt,a4paper]{report}
% Language: English
\pdfminorversion=7
\usepackage[pdftex]{graphicx}
\usepackage{changepage}
\usepackage{xcolor}
\usepackage{listings}

\lstdefinestyle{dotfiles}{
  escapeinside={(*@}{@*)}, % (*@\label{mylabel}@*)
  numbers=left,
  stepnumber=1,
  numberstyle=\tiny,
  numbersep=10pt,
  captionpos=b,
  belowcaptionskip=1\baselineskip,
  breaklines=true,
  keepspaces=true,
  columns=flexible,
  language=C,
  showstringspaces=false,
  basicstyle=\scriptsize\ttfamily,
  keywordstyle=\color{green!40!black},
  commentstyle=\itshape\color{purple!40!black},
  identifierstyle=\color{blue},
  stringstyle=\color{red},
  tabsize=2,
  morekeywords={digraph, graph, subgraph, edge, node, color, style, shape, fillcolor},
}

\newcommand{\ry}{\rotatebox{90}}
\begin{document}

\begin{titlepage}
\vspace{2cm}
\begin{center}
\Huge{Tests for the Datamaster}

\Large{Martin Skorsky}

\Large{2020-12-23}
\end{center}
\vfill
\end{titlepage}

\tableofcontents

\chapter{Overview - What is tested}
All tests are on branch \texttt{dm-fallout-tests}. The tests run with \texttt{make} or \texttt{make all} in folder \texttt{modules/ftm/tests}. 
To compile \texttt{libcarpedm} use \texttt{make prepare}. This runs \texttt{make clean} and \texttt{make} in folder \texttt{modules/ftm/ftmx86}.

Each test has a seperate target in the \texttt{Makefile}.
\begin{table}
\caption{Which test tests what}
%\begin{center}
\centering
\begin{tabular}[t]{|l|c|c|c|c|c|c|c|c|}
\hline
Test           & \ry{Tools} & \ry{libcarpedm} & \ry{firmware} & \ry{DM-Wrapper } & \ry{uses Python} & \ry{common} & \ry{make} & \ry{checks result} \\ \hline
bpcStart       &   -        &   T             &   x           &   x              &   x              &   x         &   -       &   x                \\ \hline
dmPerformance  &   x        &   T             &   x           &   -              &   -              &   -         &   x       &   -                \\ \hline
dmThreads      &   x        &   x             &   T           &   -              &   -              &   -         &   x       &   -                \\ \hline
fid7           &   -        &   x             &   T           &   x              &   x              &   x         &   -       &   x                \\ \hline
full\_test     &   -        &   T             &   x           &   -              &   x              &   -         &   -       &   x                \\ \hline
messageCounter &   -        &   x             &   x           &   T              &   x              &   -         &   -       &   -                \\ \hline
pps            &   -        &   T             &   x           &   x              &   x              &   -         &   -       &   -                \\ \hline
schedules      &   x        &   x             &   T           &   -              &   -              &   -         &   x       &   -                \\ \hline
singleEdgeTest &   -        &   T             &   -           &   -              &   -              &   -         &   -       &   x                \\ \hline
unittests      &   T        &   x             &   x           &   -              &   x              &   -         &   -       &   x                \\ \hline
\end{tabular}
%\end{center}
\end{table}
\chapter{The Tests}
\section{bpcStart}
\texttt{bpcStart} tests the implementation of the beam process chain start flag in libcarpedm. 
The test schedule sends two timing messages with \texttt{bpcstart=True} and \texttt{bpcstart=1}.
With \texttt{saft-ctl snoop} it is checked that the timing messages contain the correct setting. 
In addition with \texttt{dm-sched} the dumped schedule is checked for the bpcstart flag.
\section{dmPerformance}
\texttt{dmPerformance} tests the performance improvements in libcarpedm.
The test starts a schedule on a clean data master, checks if some part of the schedule is removable, removes it and 
then adds another schedule. This is done for a small schedule and a larger schedule. The test is ok if all commands 
work. There is no check for this.
\section{dmThreads}
\texttt{dmThreads} tests the firmware with 4 threads. The same test for 6 threads fails due to a bug in the 
firmware. For each thread a pattern with one block and one timing message is loaded.
\section{fid7}
\texttt{fid7} tests the fix for the format id 7 bug.
\section{full\_test}
\texttt{full\_test} is a collection of nine tests for szenarios with schedules, static and dynamic. The schedules are added and modified. The 
status of queues is checked. Schedules are not started.
\section{messageCounter}
\texttt{messageCounter} is a basic test for the Python-Wrapper of libcarpedm.
\section{pps}
\texttt{pps} (pulse per second) is a basic test with a schedule which sends two timing messages every second.
\section{schedules}
\texttt{schedules} collects scheduls which are started. Tests that the schedules are compiled and loaded.
\section{singleEdgeTest}
\section{singleEdgeTest}
\texttt{singleEdgeTest} tests the validation of combinations of node types and edge types. This uses libcarepedm only. There are 1936 schedules
with at least one edge and two nodes tested. Some schedule are enlarged with meta nodes.

For this, the combinations of a node and an outgoing edge and,
respectively, a node and an incoming edge.

The tables are compiled from source validation.cpp.

Classification of nodes:
\begin{enumerate}
\item Meta nodes: qinfo, qbuf, listdst
\item Event nodes: tmsg, switch, flow, flush, noop, waits
\item Command nodes: flow, flush, noop, wait
\end{enumerate}

\begin{table}
\begin{adjustwidth}{-25mm}{}
\caption{Schedule -- Valid edge types per node type}
\begin{tabular}[t]{|l|c|c|c|c|c|c|c|c|c|c|c|}

\hline
Edge Type & \multicolumn{11}{c|}{Node Type - Out-Edge, first node} \\
\hline
                 & block & blockalign & flow & flush & listdst & noop & qbuf & qinfo & switch & tmsg & wait \\
\hline
defdst           & 0..1  & 0..1       & 0..1 & 1     & --      & 0..1 & --   & --    & 0..1   & 1    & 0..1 \\
altdst           & 0..10 & 0..10      & --   & --    & --      & --   & --   & --    & --     & --   & --   \\
listdst          & 0..1  & 0..1       & --   & --    & --      & --   & --   & --    & --     & --   & --   \\
baddefdst        & --    & --         & --   & --    & --      & --   & --   & --    & --     & --   & --   \\
target (Switch)  & --    & --         & --   & --    & --      & --   & --   & --    & 0..1   & --   & --   \\
target (Command) & --    & --         & 0..1 & 0..1  & --      & 0..1 & --   & --    & --     & --   & 0..1 \\
flowdst          & --    & --         & 0..1 & --    & --      & --   & --   & --    & --     & --   & --   \\
flushovr         & --    & --         & --   & 0..1  & --      & --   & --   & --    & --     & --   & --   \\
switchdst        & --    & --         & --   & --    & --      & --   & --   & --    & 0..1   & --   & --   \\
meta             & --    & --         & --   & --    & --      & --   & --   & 2     & --     & --   & --   \\
priolo           & 0..1  & 0..1       & --   & --    & --      & --   & --   & --    & --     & --   & --   \\
priohi           & 0..1  & 0..1       & --   & --    & --      & --   & --   & --    & --     & --   & --   \\
prioil           & 0..1  & 0..1       & --   & --    & --      & --   & --   & --    & --     & --   & --   \\
dynid x          & --    & --         & --   & --    & --      & --   & --   & --    & --     & --   & --   \\
dynpar0          & --    & --         & --   & --    & --      & --   & --   & --    & --     & 0..1 & --   \\
dynpar1          & --    & --         & --   & --    & --      & --   & --   & --    & --     & 0..1 & --   \\
dyntef           & --    & --         & --   & --    & --      & --   & --   & --    & --     & --   & --   \\
dynres           & --    & --         & --   & --    & --      & --   & --   & --    & --     & --   & --   \\
any edge         & 0..15 & 0..15      & 1..3 & 1..3  & 0..     & 1..2 & 0..  & 2     & 1..3   & 1..3 & 1..2 \\
\hline
\end{tabular}
\end{adjustwidth}
\end{table}
The last row combines
\begin{enumerate}
\item the check for childless nodes (events and qinfo must have childs),
\item the sum of min and max cardinalities for the detailed edge types.
\end{enumerate}

\begin{table}
\begin{adjustwidth}{-25mm}{}
\caption{Schedule -- Valid edge types per node type}
\begin{tabular}[t]{|l|c|c|c|c|c|c|c|c|c|c|c|}

\hline
Edge Type & \multicolumn{11}{c|}{Node Type - In-Edge, second node} \\
\hline
                 & block & blockalign & flow  & flush & listdst & noop  & qbuf & qinfo & switch & tmsg & wait  \\
\hline
defdst           & 0..1  & 0..1       & 0..1  & 0..1  & --      & 0..1  & --   & --    & 0..1   & 0..1  & 0..1  \\
altdst           & 0..10 & 0..10      & 0..10 & 0..10 & --      & 0..10 & --   & --    & 0..10  & 0..10 & 0..10 \\
listdst          & 0..1  & 0..1       & --    & --    & --      & --    & --   & --    & --     & --    & --    \\
baddefdst        & --    & --         & --    & --    & --      & --    & --   & --    & --     & --    & --    \\
target (Switch)  & 0..1  & 0..1       & --    & --    & --      & --    & --   & --    & --     & --    & --    \\
target (Command) & 0..1  & 0..1       & --    & --    & --      & --    & --   & --    & --     & --    & --    \\
flowdst          & 0..1  & 0..1       & 0..1  & 0..1  & --      & 0..1  & --   & --    & 0..1   & 0..1  & 0..1  \\
flushovr         & --    & --         & --    & 0..1  & --      & --    & --   & --    & --     & --    & --    \\
switchdst        & 0..1  & 0..1       & 0..1  & 0..1  & --      & 0..1  & --   & --    & 0..1   & 0..1  & 0..1  \\
meta             & --    & --         & --    & --    & --      & --    & 1    & --    & --     & --    & --    \\
priolo           & --    & --         & --    & --    & --      & --    & --   & 0..1  & --     & --    & --    \\
priohi           & --    & --         & --    & --    & --      & --    & --   & 0..1  & --     & --    & --    \\
prioil           & --    & --         & --    & --    & --      & --    & --   & 0..1  & --     & --    & --    \\
dynid x          & --    & --         & --    & --    & --      & --    & --   & --    & --     & --    & --    \\
dynpar0          & 0..1  & 0..1       & 0..1  & 0..1  & --      & 0..1  & --   & --    & 0..1   & 0..1  & 0..1  \\
dynpar1          & 0..1  & 0..1       & 0..1  & 0..1  & --      & 0..1  & --   & --    & 0..1   & 0..1  & 0..1  \\
dyntef           & --    & --         & --    & --    & --      & --    & --   & --    & --     & --    & --    \\
dynres           & --    & --         & --    & --    & --      & --    & --   & --    & --     & --    & --    \\
any edge         & 0..18 & 0..18      & 0..15 & 0..16 & 0       & 0..15 & 1    & 0..1  & 0..15  & 0..15 & 0..15 \\
\hline
\end{tabular}
\end{adjustwidth}
\end{table}

Open questions:
\begin{enumerate}
\item Is the list of node types complete?
\item Is the list of edge types complete?
\item Node type listdst: no rules for in- or out-edges defined. Is this correct?
\item Edge types dynid x, dyntef, dynres: are these edge types used?
\item How to distinguish edge types target (Switch) and target (Command)?
\item Node type flush and flow: these nodes may have priority queues (lo, hi, il). Why are the queues allowed?
There are no meta nodes generated for these queues. Why?
\item The rules allow two edges from a block to a bufferlist with edge types priolo and priohi.
Should the rules be fixed for this invalid schedule?
\item Is an edge of type defdst necessary for a node of type flow? For node type flush it is necessary.
\item There are no ConstellationRules for nodes of type listdst and qbuf. Is this correct? Yes! Node types must be childless.
\item Nodes types listdst and qbuf must be childless.
\end{enumerate}


\section{unittests}
\texttt{unittests} contains Python test scripts which contain Python unit tests for the tools dm-cmd, dm-sched. Each unit test calls dm-cmd 
with commands and options and checks the result with the output on stdout and stderr. There are also negative tests with an invalid command 
line. These tests are successful when the response is the correct error message and not a core dump.
\chapter{Common Components}
\section{dm\_testbench.py}
\texttt{dm\_testbench.py} is a collection of Python functions for use in other test scripts.
\begin{enumerate}
\item startpattern(data\_master, pattern\_file)
    Connect to the given data master and load the pattern file (dot format).
    The data master is halted, cleared, and statistics is reset.
    Search for the first pattern in the data master with 'dm-sched' and start it.
\end{enumerate}
\end{document}
