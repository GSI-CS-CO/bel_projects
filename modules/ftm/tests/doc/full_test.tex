\section{full\_test}
\texttt{full\_test} is a collection of nine tests for szenarios with schedules, static and dynamic. The schedules are added and modified. The
status of queues is checked. Schedules are not started.
\subsection{How to run the tests}
Currently there are 9 automated tests for the data master. The tests are driven by a Python script \texttt{dm\_testman.py}.

The static tests just add and remove patterns. The dynamic test in addition start the patterns.
\subsection{Prerequisites}
This test framework requires Python 3.6m or higher. A data master has to be accesible from the host. 
The source for the tests is at 

Project: \texttt{https://github.com/GSI-CS-CO/bel\_projects/},

Path: \texttt{modules/ftm/ftmx86/full\_test/}.

For each test the data master is halted and cleared. All patterns are removed.

\subsection{Structure of Description}
\begin{enumerate}
	\item Purpose of Test

	What is the objective of this test?
	\item Prerequisites of Test

	What is the setting of the test?
	\item Test Actions

	List the actions of the test. This includes the graphs of the test pattern.
	\item Success Criteria

	What is checked to state a successful test?
\end{enumerate}
\subsection{Static Tests - Basic}
\begin{enumerate}
	\item Purpose of Test

	This test uses \texttt{dm-sched add} and \texttt{dm-sched remove}. The pattern Figure~\ref{fig:Pattern_for_the_static_basic_test} 
	is loaded into data master and removed afterwards.
	\item Test Actions

	On a cleared data master the test pattern is added with \texttt{dm-sched add}. With \texttt{dm-sched status} it is checked that 24 nodes with the expected names are available. The test pattern is removed with \texttt{dm-sched remove}. At the end \texttt{dem-sched status} is used to check that no pattern is present on the data master.

    \begin{figure}
        \centering 
        \includegraphics{TestPattern/static_basic.pdf}
        \caption{Pattern for the static basic test}
        \label{fig:Pattern_for_the_static_basic_test}
    \end{figure}
	\item Success Criteria

	The test is successful if no pattern is loaded. Checked with \texttt{dm-sched status}
\end{enumerate}
\subsection{Static Tests - Coupling}
\begin{enumerate}
	\item Purpose of Test

	This test enlarges an existing pattern with a second pattern with edges into the first 
	pattern. See Figure~\ref{fig:Pattern_for_the_static_coupling_test} for the test patterns.
	\item Test Actions

	First, a pattern with three nodes is added. In a second step a pattern with additional three nodes is added. 
	This pattern contains edges into the first pattern.
    \begin{figure}
        \centering 
        \includegraphics{TestPattern/static_coupling1.pdf}
        \includegraphics{TestPattern/static_coupling2.pdf}
        \caption{Pattern for the static coupling test before and after coupling}
        \label{fig:Pattern_for_the_static_coupling_test}
    \end{figure}
	\item Success Criteria

	After adding the two patterns the status is checked with \texttt{dm-sched status}. The resulting \texttt{download.dot} is 
	compared to an expected dot-file.
\end{enumerate}
\subsection{Static Tests - Priority and Type}
\begin{enumerate}
	\item Purpose of Test

The test checks the relative and the absolute time values for two nodes in a four node pattern. 
See Figure~\ref{fig:Pattern_for_the_static_priority_and_type_test} for the test pattern 
and \ref{fig:Pattern_for_the_static_priority_and_type_test_with_meta_nodes} for the test pattern with meta nodes, 
displaying the priority queues.
	\item Test Actions

	Add the pattern, check the relative time values. Clear the data master. Add the pattern again and check the absolute time values.
    \begin{figure}
        \centering 
        \includegraphics{TestPattern/static_prio_and_type.pdf}
        \caption{Pattern for the static priority and type test}
        \label{fig:Pattern_for_the_static_priority_and_type_test}
    \end{figure}
    \begin{figure}
        \centering 
        \includegraphics*[height=0.95\textheight,keepaspectratio]{TestPattern/static_prio_and_type_meta.pdf}
        \caption{Pattern for the static priority and type test with meta nodes}
        \label{fig:Pattern_for_the_static_priority_and_type_test_with_meta_nodes}
    \end{figure}
	\item Success Criteria

	Two checks of the time values with \texttt{dm-cmd rawqueue}.
\end{enumerate}
\subsubsection{Dynanamic Tests - Async}
\begin{enumerate}
	\item Purpose of Test

	For a schedule asynchronous clear a block, change the destination to a timing message and check that all nodes are visited.

	See Figure~\ref{fig:Pattern_for_the_dynamic_async_test} for the test pattern.
	\item Test Actions

	Upload test schedule and start pattern $LOOP$. Check with \texttt{dm-cmd rawvisited}. The 
	nodes 'BLOCK\_B', 'BLOCK\_LOOP', 'CMD\_LOOP' are visited. Lock pattern 'B' (this locks 'BLOCK\_B'), check this with 
	\texttt{showlocks}. Clear pattern 'B' which clears the queues of 'BLOCK\_B', check this with \texttt{rawqueue}. 
	Change schedule with flow command. Destination of 'BLOCK\_B' is now 'MSG\_A' for one message. Check this 
	with \texttt{rawqueue}. Unlock pattern 'B', wait for $1.2$ seconds and then check with \texttt{rawvisited} that all 
	nodes are visited and no blocks are locked (with \texttt{showlocks}).
    \begin{figure}
        \centering 
        \includegraphics{TestPattern/dynamic_async.pdf}
        \caption{Pattern for the dynamic async test}
        \label{fig:Pattern_for_the_dynamic_async_test}
    \end{figure}
	\item Success Criteria
	
	Check that node 'MSG\_A' is visited after changing the destination of the flow.
\end{enumerate}
\subsubsection{Dynamic Tests - Basic - Run CPU 0 single}
\begin{enumerate}
	\item Purpose of Test

    Add the test schedule to the data master, start patterns and check which nodes were visited.

	See Figure~\ref{fig:Pattern_for_the_dynamic_run_CPU_0_single_test} for the test pattern.
	\item Test Actions
    
    First check that no block is visited. Start pattern 'IN0'. Check that 'BLOCK\_A' and 'BLOCK\_IN0' are visited. 
    Start pattern 'IN1'. Check that in addition 'BLOCK\_B' and 'BLOCK\_IN1' are visited.
    \begin{figure}
        \centering 
        \includegraphics{TestPattern/dynamic_basic_run_cpu0_single.pdf}
        \caption{Pattern for the dynamic run CPU 0 single test}
        \label{fig:Pattern_for_the_dynamic_run_CPU_0_single_test}
    \end{figure}
	\item Success Criteria

	In the end all blocks are visited.
\end{enumerate}
\subsubsection{Dynamic Tests - Basic - Run all single}
\begin{enumerate}
	\item Purpose of Test

    Run a very basic schedule on all four CPUs.

	See Figure~\ref{fig:Pattern_for_the_dynamic_run_all_test} for the test pattern.
	\item Test Actions

	Add a schedule with four blocks to the data master. For each CPU, check that no block is visited. Then start 
	a pattern for one CPU and check that the block specific for this pattern is visited.
    \begin{figure}
        \centering 
        \includegraphics{TestPattern/dynamic_basic_run_all_single.pdf}
        \caption{Pattern for the dynamic run all test}
        \label{fig:Pattern_for_the_dynamic_run_all_test}
    \end{figure}
	\item Success Criteria

	For each pattern the correct CPU is used.
\end{enumerate}
\subsubsection{Dynamic Tests - Start Stop Abort}
\begin{enumerate}
	\item Purpose of Test

    First part: Start and abort a pattern. Second part: Start and stop a pattern.

	See Figure~\ref{fig:Pattern_for_the_dynamic_start_stop_abort_test} for the test pattern.
	\item Test Actions
    
    Add a schedule, check that CPU 0 is idle and then start pattern 'IN\_C0'. Check that the pattern is running. 
    Abort the pattern 'IN\_C0'. Check the visited nodes for the pattern. The second part is similar. 
    Add the same schedule, check that CPU 0 is idle and then start pattern 'IN\_C0'. Check that the pattern is running. 
    Stop the pattern 'IN\_C0'. Check the visited nodes for the pattern.
    \begin{figure}
        \centering 
        \includegraphics{TestPattern/dynamic_basic_start_stop_abort.pdf}
        \caption{Pattern for the dynamic start stop abort test}
        \label{fig:Pattern_for_the_dynamic_start_stop_abort_test}
    \end{figure}
	\item Success Criteria

	Check that pattern is correctly aborted (rawstatus RUN is 0 immediately) or 
	stopped (rawstatus RUN is 1 immediately, but 0 after 1.5 seconds).
\end{enumerate}
\subsubsection{Dynamic Tests - Branch - Single}
\begin{enumerate}
	\item Purpose of Test

	Test that the flow command switches from one block to another.

	See Figure~\ref{fig:Pattern_for_the_dynamic_branch_single_test} for the test pattern.
	\item Test Actions

	Add a schedule, start the pattern 'IN\_C0'. After checking that nodes 'BLOCK\_IN0' and 'BLOCK\_A' are visited, 
        change the flow with the flowpattern command at pattern 'IN\_C0' from 'A' to 'B'. Check that the flowpattern 
        command is in the low priority queue and then start the pattern 'IN\_C0'. Check that the flowpattern 
        command is processed in the low priority queue and node 'BLOCK\_B' is visited.
    \begin{figure}
        \centering 
        \includegraphics{TestPattern/dynamic_branch_single.pdf}
        \caption{Pattern for the dynamic branch single test}
        \label{fig:Pattern_for_the_dynamic_branch_single_test}
    \end{figure}
	\item Success Criteria

	Changing flow from 'BLOCK\_A' to 'BLOCK\_B' works.
\end{enumerate}
\subsubsection{Dynamic Tests - Branch - Coupling}
\begin{enumerate}
	\item Purpose of Test

	Test not working, needs set up.
	\item Test Actions
	\item Success Criteria
\end{enumerate}
\subsubsection{Dynamic Tests - Loop}
\begin{enumerate}
	\item Purpose of Test

        Test loop with flow initialiser.

	See Figure~\ref{fig:Pattern_for_the_dynamic_loop_test} for the test pattern.
	\item Test Actions

        Add a schedule and start pattern 'IN\_A'. Check the visited nodes. Nodes 'INIT\_A0', 'BLOCK\_LOOP', and 'BLOCK\_EXIT' 
        should be visited. Start pattern 'IN\_B' and check the visited nodes. All nodes should be visited.
    \begin{figure}
        \centering 
        \includegraphics{TestPattern/dynamic_loop.pdf}
        \caption{Pattern for the dynamic loop test}
        \label{fig:Pattern_for_the_dynamic_loop_test}
    \end{figure}
	\item Success Criteria

	In the end all blocks are visited.
\end{enumerate}
\subsubsection{Dynamic Tests - Switch}
\begin{enumerate}
	\item Purpose of Test

	Test not working, needs set up.
	\item Test Actions
	\item Success Criteria
\end{enumerate}
%\end{document}
